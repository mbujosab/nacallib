\RequirePackage{color}
\RequirePackage{amsmath}
\RequirePackage{amssymb}
\RequirePackage{xspace}
\RequirePackage{amsfonts}
\RequirePackage[mathscr]{eucal}
\RequirePackage{amsxtra}
\RequirePackage{mathtools} %para \prescript
\RequirePackage{ifthen} 
\RequirePackage{xifthen} %para \isempty en \TESC


%%%%%%%%%%%%%%%%%%%%%%%%%%%%%%%%%%%%%%%%%%%%%%
%%%%%%%%%%%%%%%%%%%%%%%%%%%%%%%%%%%%%%%%%%%%%%
%%                NOTACION                  %%
%%%%%%%%%%%%%%%%%%%%%%%%%%%%%%%%%%%%%%%%%%%%%%
%%%%%%%%%%%%%%%%%%%%%%%%%%%%%%%%%%%%%%%%%%%%%%

\def\MDaV{D}   %%% Matriz Diagonal de Autovalores \Lambda

%%%%%%%%%%%%%%%%%%%%%%%%%%%%%%%
% Muy generica (e inecesaria....)
%%%%%%%%%%%%%%%%%%%%%%%%%%%%%%%

\newcommand{\funcion}[2]{\ensuremath{{#1}_{#2}}\xspace}

%%%%%%%%%%%%%%%%%%%%%%%%%%%%%%%
%%% Notación Matricial
%%%%%%%%%%%%%%%%%%%%%%%%%%%%%%

%% Notación para la matriz traspuesta
%\newcommand{\T}{\prime} % Notaci^^f3n para la matriz traspuesta
%\newcommand{\T}{\textnormal{\!\tiny{T}\!\!}} 
\newcommand{\T}{\ensuremath{\intercal}} 
%\DeclareRobustCommand{\transp}{^{\rm T}}

\newenvironment{m}[1]{\left[\begin{array}{@{\,}#1@{\,}}}{\end{array}\right]}
%% Matriz #1
\newenvironment{p}[1]{\left(\begin{array}{@{\,}#1@{\,}}}{\end{array}\right)}
%% Matriz #1

% conjugado
%\newcommand{\conj}[1]{\ensuremath{\accentset{\star}{#1}}\xspace}

\newcommand\widebar[1]{\mathop{\overline{#1}}}
\newcommand{\conj}[1]{\ensuremath{\widebar{#1}}\xspace} 


%%%%%%%%%%%%%%%%%%%%%%%%%%%%%%%%%% MATRICES Y VECTORES %%%%%%%%%%%%%%%%%

\RequirePackage{vector}
%\RequirePackage{accents}

%%% MATRIZ
\newcommand{\mat}[1]{\ensuremath{\boldsymbol{\svec{\MakeUppercase{#1}}}}\xspace} 
\newcommand{\Mat}[2][{}]{\ensuremath{{{\mat{#2}}_{#1}}}\xspace} 
\newcommand{\SubMat}[2][{}]{\ensuremath{{{\mat{#2}_{\boldsymbol{#1}}}}}\xspace} 

%% sistema de vectores
\newcommand{\SV}[2][{}]{\ensuremath{ {\mathsf{\MakeUppercase{#2}}}_{#1} }\xspace}


%%%%% VECTOR de Rn
%\renewcommand{\uvec}[1]{{#1}} %%%%% DESCOMENTAR ESTO SUPONE QUITAR EL SUBRAYADO DE LOS VECTORES
\newcommand{\vect} [2][{}]{\ensuremath{{\vec{\MakeLowercase{#2}}}_{#1}\xspace}}

%\newcommand{\Vect} [2][{}]{\ensuremath{{\uvec{{\boldsymbol{\MakeLowercase{#2}}}_{#1}}}\xspace}}
\newcommand{\Vect} [2][{}]{\ensuremath{{{{\boldsymbol{\MakeLowercase{#2}}}_{#1}}}\xspace}}


%%%%%%%%%%%%%%%%%%%%%%%%%%
%%% OJO NO SÉ SI TIENE SENTIDO ya...
%\newcommand{\MVect}[2][{}]{\ensuremath{\left[{\uvec{{\boldsymbol{\MakeLowercase{#2}}}_{#1}}}\vphantom{\Big(}\right]  \xspace}}
\newcommand{\MVect}[2][{}]{\ensuremath{\left[{{{\boldsymbol{\MakeLowercase{#2}}}_{#1}}}\vphantom{\Big(}\right]  \xspace}}

%%%%%%%%% OJO NO TIENEN SENTIDO.... transposición de vectores!!!!!!!!!!!!
\newcommand{\VectT}[2][{}]{\ensuremath{{\uvec{{\boldsymbol{\MakeLowercase{#2}}}_{#1}^\T\!}\,}\xspace}} %%% OJO!
%\renewcommand{\VectT}[2][{}]{\ensuremath{\left[{\uvec{{\boldsymbol{\MakeLowercase{#2}}}_{#1}}}\vphantom{\Big(}\right]^{\!\T}\xspace}} %% OJO!
\renewcommand{\VectT}[2][{}]{\ensuremath{\left[{{{\boldsymbol{\MakeLowercase{#2}}}_{#1}}}\vphantom{\Big(}\right]^{\!\T}\xspace}} %% OJO!


%%% SÍMBOLO PARA INDICAR QUE SE SELECCIONA UN ÍNDICE 
%\newcommand{\getItem}{\ensuremath{\boldsymbol{\mid}}\xspace}
\newcommand{\getItem}{\ensuremath{\pmb{\mid}}\xspace}
%\newcommand{\getItem}{\ensuremath{\wr}\xspace}
%\newcommand{\getItem}{:}  %% :
%\newcommand{\getItem}{\ddagger}  
%\newcommand{\getItem}{\vdotdot} % \newcommand{\getItem}{\hdotdot} %% : % requiere del paquete fdsymbol que modifica otros símbolos
%\newcommand{\getItem}{\bullett} %\newcommand{\getItem}{\dblcolon}  %% :: %\newcommand{\getItem}{\smallblackstar} %% estrella negra

%%% SÍMBOLO PARA INDICAR SELECCIÓN DE ÍNDICE POR LA IZQUIERDA O POR LA DERECHA 
\newcommand{\getitemL}[1]{{#1}\getItem}
\newcommand{\getitemR}[1]{\getItem{#1}}
%%% SELECCIÓN CON SUBÍNDICE POR LA IZQUIERDA (FILAS), POR LA DERECHA (COLUMNAS), O AMBOS
\newcommand{\elemL}[2]{\ensuremath{\prescript{\phantom{\T}}{\getitemL{#2}}{#1}}\xspace} 
\newcommand{\elemR}[2]{\ensuremath{{#1}^{\phantom{\T}}_{\getitemR{#2}}}\xspace} 
\newcommand{\elemLP}[2]{\ensuremath{\prescript{\phantom{\T}}{\getitemL{#2}}{\big(#1\big)}}\xspace} 
\newcommand{\elemRP}[2]{\ensuremath{{\big(#1\big)}^{\phantom{\T}}_{\getitemR{#2}}}\xspace} 
\newcommand{\elemLPE}[2]{\ensuremath{\big(\prescript{\phantom{\T}}{\getitemL{#2}}{#1}\big)}\xspace} 
\newcommand{\elemRPE}[2]{\ensuremath{\big({#1}^{\phantom{\T}}_{\getitemR{#2}}\big)}\xspace} 
\newcommand{\elemLR}[3]{\ensuremath{\prescript{\phantom{\T}}{\getitemL{#2}}{#1}^{\phantom{\T}}_{\getitemR{#3}}}\xspace}

\newcommand{\fueraitemL}[1]{{#1}^{\!\Lsh}_{\vphantom{T}}}
\newcommand{\fueraitemR}[1]{\prescript{\Rsh}{\vphantom{T}}{\!\!#1}}

\newcommand{\quitaLR}[3]{\ensuremath{\prescript{\fueraitemL{#2}}{\vphantom{\Big(}}{#1}^{\fueraitemR{#3}}_{\vphantom{\Big(}}}\xspace}
%\newcommand{\quitaL}[2]{\ensuremath{\prescript{\fueraitemL{#2}}{\vphantom{\Big(}}{#1}}\xspace}

\newcommand{\quitaL}[2]{\ensuremath{\prescript{\fueraitemL{#2}}{\vphantom{(}}{\!#1}}\xspace}

\newcommand{\quitaR}[2]{\ensuremath{{#1}^{\!\fueraitemR{#2}}_{\vphantom{(}}}\xspace}


\newcommand{\subMat}[3]{\ensuremath{\prescript{\fueraitemL{#2}}{\vphantom{(}}{\!\Mat{#1}}^{\!\fueraitemR{#3}}_{\vphantom{(}}}\xspace}
%\renewcommand{\subMat}[3]{\ensuremath{M_{#2#3}\big(\Mat{#1}\big)}\xspace}
\newcommand{\Menor}[3]{\ensuremath{\det\left(\subMat{#1}{#2}{#3}\right)}\xspace}
\newcommand{\MenoR}[3]{\ensuremath{\left|\subMat{#1}{#2}{#3}\right|}\xspace}

\newcommand{\Cof}[3]{\ensuremath{\cof\Big(\eleM{#1}{#2}{#3}\Big)}\xspace}
\newcommand{\Coff}[3]{\ensuremath{\cof(\eleMM{#1}{#2}{#3})}\xspace}


\newcommand{\TrEE}{\ensuremath{\pmb{\tau}}\xspace}
\newcommand{\TrE}[1]{\ensuremath{\underset{\left[#1\right]}{\TrEE}}\xspace}

\newcommand{\pe}[2]{ \pmb{#1} \rightleftharpoons \pmb{#2} }  % intercambio
\newcommand{\pr}[2]{ \left({#1}\right){\pmb{#2}}                }  % producto
\newcommand{\su}[3]{ \left({#1}\right){\pmb{#2}} + {\pmb{#3}}   }  % suma
%\newcommand{\pr}[2]{ {\left(#1\right)}{\pmb{#2}} }                 % producto
%\newcommand{\su}[3]{ {\left(#1\right)}{\pmb{#2}}+{\pmb{#3}} }         % suma

%%% TRANSFORMACIONES ELEMENTALES GENÉRICAS
%% suceción de transformaciones elementales genéricas desde #1 hasta #2
\newcommand{\SSTEC}[3]{  \ensuremath{                    {#3}_{\TrEE_{#1}\cdots\TrEE_{#2}}^{\phantom{\T}} }\xspace }
\newcommand{\SSTEF}[3]{  \ensuremath{ \prescript{\phantom{\T}}{\TrEE_{#2}\cdots\TrEE_{#1}}{#3}            }\xspace }
\newcommand{\SSTEFC}[3]{ \ensuremath{ \prescript{\phantom{\T}}{\TrEE_{#2}\cdots\TrEE_{#1}}{#3}_{\TrEE_{#1}\cdots\TrEE_{#2}}^{\phantom{\T}} }\xspace }

\usepackage{tensor}

\newcommand{\STEC}[2][]{%                       %% Ejemplo: \STEC[ \TrE{\su{i}{j}{+5}} \TrE{\pr{j}{-7}} ]{\Mat{A}}
  \ifthenelse{\isempty{#1}}%
    { \SSTEC{1}{k}{#2} }                                   % if #1 is empty   Suceriones de la 1 a la k
    { \ensuremath{ {#2}_{#1}^{\phantom{\T}} } }  
    \xspace
}

\renewcommand{\STEC}[2][]{%                       %% Ejemplo: \STEC[ \TrE{\su{i}{j}{+5}} \TrE{\pr{j}{-7}} ]{\Mat{A}}
  \ifthenelse{\isempty{#1}}%
    { \SSTEC{1}{k}{#2} }                                   % if #1 is empty   Suceriones de la 1 a la k
    { \ensuremath{ \tensor*[_{\vphantom{#1}}^{\vphantom{\T}}]{#2}{_{#1}^{\phantom{\T}}} } }  
    \xspace
}

\newcommand{\STEF}[2][]{%                       %% Ejemplo: \STEF[ \TrE{\su{i}{j}{+5}} \TrE{\pr{j}{-7}} ]{\Mat{A}}
  \ifthenelse{\isempty{#1}}%
    { \SSTEF{1}{k}{#2} }                                   % if #1 is empty
    { \ensuremath{ \prescript{\phantom{\T}}{_{#1}}{#2} } } 
    \xspace
}

\renewcommand{\STEF}[2][]{%                       %% Ejemplo: \STEF[ \TrE{\su{i}{j}{+5}} \TrE{\pr{j}{-7}} ]{\Mat{A}}
  \ifthenelse{\isempty{#1}}%
    { \SSTEF{1}{k}{#2} }                                   % if #1 is empty
    { \ensuremath{ \tensor*[^{\phantom{\T}}_{#1}]{#2}{^{\vphantom{\T}}_{\vphantom{#1}}} } } 
    \xspace
}

\newcommand{\STEFC}[3]{%   Por los dos lados!   %% Ejemplo: \STEFC{\TrE{\su{i}{j}{-5}}}{\TrE{\pr{j}{-7}}}{\Mat{A}}
    \ensuremath{\prescript{\phantom{\T}}{#1}{{#3}}^{\phantom{\T}}_{#2}} 
    \xspace                                          
}   


\newcommand{\TEC}[2][]{%
  \ifthenelse{\isempty{#1}}%
    {\STEC[\TrEE]{#2}}                 % if #1 is empty   A_tau
    {\STEC[\TrEE_{#1}]{#2}}            %                  A_{tau_{#1}}
    \xspace
}

\newcommand{\TEF}[2][]{%
  \ifthenelse{\isempty{#1}}%
    {\STEF[\TrEE]{#2}}                 % if #1 is empty
    {\STEF[\TrEE_{#1}]{#2}}
    \xspace
}

%% matricial

% SUMAS
\newcommand{\TESC}[4]{\STEC[\!\TrE{\su{#1}{#2}{#3}}]{#4}}
\newcommand{\TESF}[4]{\STEF[\TrE{\su{#1}{#2}{#3}}\!]{#4}}


% con paréntesis
\newcommand{\TESCP}[4]{\ensuremath{{\left({#4}\right)}^{\phantom{\big(}}_{\!\!\!\TrE{\su{#1}{#2}{#3}}}}\xspace}
\newcommand{\TESFP}[4]{\ensuremath{\prescript{\phantom{\big(}}{\TrE{\su{#1}{#2}{#3}}\!\!}{\left({#4}\right)}}\xspace}

% PRODUCTOS
\newcommand{\TEPC}[3]{\STEC[\!\TrE{\pr{#1}{#2}}]{#3}}
\newcommand{\TEPF}[3]{\STEF[\TrE{\pr{#1}{#2}}\!]{#3}}
% con paréntesis
\newcommand{\TEPCP}[3]{\ensuremath{{\left({#3}\right)}^{\phantom{\big(}}_{\!\!\!\TrE{\pr{#1}{#2}}}}\xspace}
\newcommand{\TEPFP}[3]{\ensuremath{\prescript{\phantom{\big(}}{\TrE{\pr{#1}{#2}}\!\!}{\left({#3}\right)}}\xspace}

% INTERCAMBIOS
\newcommand{\TEIF}[3]{\STEF[\TrE{\pe{#1}{#2}}\!]{#3}}
\newcommand{\TEIC}[3]{\STEC[\!\TrE{\pe{#1}{#2}}]{#3}}
% con paréntesis
\newcommand{\TEIFP}[3]{\ensuremath{\prescript{\phantom{\big(}}{\TrE{\pe{#1}{#2}}\!\!}{\left({#3}\right)}}\xspace}
\newcommand{\TEICP}[3]{\ensuremath{{\left({#3}\right)}^{\phantom{\big(}}_{\!\!\!\TrE{\pe{#1}{#2}}}}\xspace}

\newcommand{\TESCC}[3]{\ensuremath{c(\su{#1}{#2}{#3})}\xspace}
\newcommand{\TESFF}[3]{\ensuremath{\eng{f}{f}(\su{#1}{#2}{#3})}\xspace}
\newcommand{\TEPCC}[2]{\ensuremath{c(\pr{#1}{#2})}\xspace}
\newcommand{\TEPFF}[2]{\ensuremath{\eng{f}{f}(\pr{#1}{#2})}\xspace}
\newcommand{\TEICC}[2]{\ensuremath{c(\pe{#1}{#2})}\xspace}
\newcommand{\TEIFF}[2]{\ensuremath{\eng{f}{f}(\pe{#1}{#2})}\xspace}


% transformaciones por los dos lados con paréntesis (no creo que lo use...)
%\newcommand{\TESFCP}[7]{\ensuremath{\prescript{\phantom{\big(}}{\TrE{\su{#1}{#2}{#3}}\!\!}{\left({#4}\right)}^{\phantom{\big(}}_{\!\!\TrE{\su{#5}{#6}{#7}}}}\xspace}
%\newcommand{\TEPFC}[5]{\ensuremath{\prescript{\phantom{\big(}}{\TrE{\pr{#1}{#2}}\!\!}{\left({#3}\right)}^{\phantom{\big(}}_{\!\!\TrE{\pr{#4}{#5}}}}\xspace}
\newcommand{\TEIFC}[5]{\ensuremath{\prescript{\phantom{\big(}}{\TrE{\pe{#1}{#2}}\!\!}{\left({#3}\right)}^{\phantom{\big(}}_{\!\!\TrE{\pe{#4}{#5}}}}\xspace}

%% funcional (que no uso)
%\newcommand{\TEESC}[4]{\ensuremath{\TECI{#1}{#2}{#3}(\Mat{#4})}\xspace}
%\newcommand{\TEESF}[4]{\ensuremath{\TEFI{#1}{#2}{#3}(\Mat{#4})}\xspace}

%% MATRICES ELEMENTALES

\newcommand{\EM}[2]{\ensuremath{\Mat{E_{\MakeLowercase{[#1,#2]}}}}}
\newcommand{\InvEM}[2]{\ensuremath{\Mat{E^{-1}_{\MakeLowercase{[#1,#2]}}}}}

%\newcommand{\EI}[3]{\ensuremath{\boldsymbol{\svec{E}}_{\boldsymbol{{[#1,#2]}}}^{\boldsymbol{\svec{\mathit{I}}}\left(#3\right)}}}
%\newcommand{\EII}[2]{\ensuremath{\boldsymbol{\svec{E}}_{\boldsymbol{{[#1,#1]}}}^{\boldsymbol{\svec{\mathit{II}}}\left(#2\right)}}}

\newcommand{\EI}[3]{\ensuremath{\boldsymbol{\svec{E}}_{\boldsymbol{{[#2,#3]}}}^{\left(#1\right)}}}
\newcommand{\EII}[2]{\ensuremath{\boldsymbol{\svec{E}}_{\boldsymbol{{[#2,#2]}}}^{\left(#1\right)}}}

\newcommand*\circled[1]{\tikz[baseline=(char.base)]{\node[shape=circle,draw,inner sep=2pt] (char) {#1};}}

%\DeclareRobustCommand{\MyCommand}

\newcommand{\Mint}[2]{\ensuremath{\underset{\boldsymbol{#1}\rightleftarrows\boldsymbol{#2}}{\Mat{P}}}\xspace}

\newcommand{\MP}[1][]{%
  \ifthenelse{\isempty{#1}}%
    {\ensuremath{\Mat{P_{\!\!_\circlearrowleft}}\!}}  % if #1 is empty
    {\ensuremath{{\underset{#1}{\overset{\rightleftarrows}{\Mat{P}}}}}}
    \xspace
}


% \newcommand{\EI}[3]{\ensuremath{\Mat{E^\mathit{I}_{\MakeLowercase{#1#2}}}(#3)}}
% \newcommand{\EII}[2]{\ensuremath{\Mat{E^\mathit{II}_{\MakeLowercase{#1#1}}}(#2)}}
% \newcommand{\EI}[3]{\ensuremath{\Mat{E^{\mathit{I}(#3)}_{\MakeLowercase{#1#2}}}}}
% \newcommand{\EII}[2]{\ensuremath{\Mat{E^{\mathit{II}(#2)}_{\MakeLowercase{#1#1}}}}}

%%\newcommand{\EI}[3]{\ensuremath{\Mat{E^{\mathit{I}(#3)}_{\MakeLowercase{[#1,#2]}}}}}
%%\newcommand{\EII}[2]{\ensuremath{\Mat{E^{\mathit{II}(#2)}_{\MakeLowercase{[#1,#1]}}}}}

%\newcommand{\EI}[3]{\ensuremath{\Mat{E^{(#3)}_{\MakeLowercase{#1#2}}}}}
%\newcommand{\EII}[2]{\ensuremath{\Mat{E^{(#2)}_{\MakeLowercase{#1#1}}}}}
%\newcommand{\MP}[1]{\ensuremath{\Mat{P_{\MakeLowercase{#1}}}}}


%\renewcommand{\MP}[2][]{%
%  \ifthenelse{\isempty{#1}}%
%    {\ensuremath{\Mat{P_{\!\!_\circlearrowleft}}\!}}  % if #1 is empty
%    {\ensuremath{\underset{#1\leftrightarrow #2}{\Mat{P}}}}
%    \xspace
%}

    %{\ensuremath{\left[\underset{#1}{\overset{\rightleftarrows}{\Mat{P}}}\right]}}
    %{\ensuremath{\overset{\circlearrowleft}{\Mat{P}}}}
    %{\ensuremath{\Mat{P}_{\!\!_\MakeLowercase{#1}}}}% if #1 is not empty
    %{\ensuremath{\Mat{P_{\boldsymbol{\MakeLowercase{#1}}}}}}% if #1 is not empty




%\newcommand{\TEECI}[4]{\ensuremath{\TECI{#1}{#2}{#3}(\Mat{#4})}\xspace}
%\newcommand{\TEEECII}[3]{\ensuremath{\TECI{#1}{#2}(\Mat{#3})}\xspace}
%\newcommand{\TEIIL}[3]{\ensuremath{\prescript{\phantom{\T}}{\TrE{\pr{#2}{#3}}\!\!}{\Mat{#1}}}\xspace}
%\newcommand{\TEIIR}[3]{\ensuremath{{\Mat{#1}}^{\phantom{\T}}_{\!\!\TrE{\pr{#2}{#3}}}}\xspace}
%\newcommand{\TEIILR}[5]{\ensuremath{\prescript{\phantom{\T}}{\TrE{\pr{#2}{#3}}\!\!}{\Mat{#1}}^{\phantom{\T}}_{\!\!\TrE{\pr{#2}{#3}}}}\xspace}
%\newcommand{\TEIIIL}[3]{\ensuremath{\prescript{\phantom{\T}}{\TrE{\pe{#2}{#3}}\!\!}{\Mat{#1}}}\xspace}
%\newcommand{\TEIIIR}[3]{\ensuremath{{\Mat{#1}}^{\phantom{\T}}_{\!\!\TrE{\pe{#2}{#3}}}}\xspace}
%\newcommand{\TEIIILR}[5]{\ensuremath{\prescript{\phantom{\T}}{\TrE{\pe{#2}{#3}}\!\!}{\Mat{#1}}^{\phantom{\T}}_{\!\!\TrE{\pe{#2}{#3}}}}\xspace}

%%% SELECCIÓN CON PARÉNTESIS POR DETRÁS (fila,columna) o solo (elemento) para vectores
\newcommand{\elemUU}[2]{\ensuremath{{#1}(#2)}\xspace}                            % este solo para vectores!
\newcommand{\elemUUU}[2]{\ensuremath{\textrm{elem}_{#2}(#1)}\xspace}             % este solo para vectores!
\newcommand{\elemRR}[2]{\ensuremath{{#1}({\getItem},#2)}\xspace} 
\newcommand{\elemLL}[2]{\ensuremath{{#1}(#2,{\getItem})}\xspace} 
\newcommand{\elemLLRR}[3]{\ensuremath{{#1}(#2,#3)}\xspace}
\newcommand{\elemLLL}[2]{\ensuremath{\textrm{\eng{fila}{row}}_{#2}{#1}}\xspace}  % con texto "fila"
\newcommand{\elemRRR}[2]{\ensuremath{\textrm{col}_{#2}{#1}}\xspace}              % con texto "col"
\newcommand{\elemLLLRRR}[3]{\ensuremath{\textrm{elem}_{#2#3}{#1}}\xspace}        % con texto "elem"

%SELECCIONA UN ELEMENTO DE UNA MATRIZ O DE UNA MATRIZ TRANSPUESTA
\newcommand{\eleM}[3]{\ensuremath{\elemLR{\Mat{#1}}{#2}{#3}}\xspace}          % con subindices
\newcommand{\eleMT}[3]{\ensuremath{\elemLR{\Mat{#1}^\T\!}{#2}{#3}}\xspace}    % con subindices
%\newcommand{\eleMM}[3]{\ensuremath{\elemLLRR{\Mat{#1}}{#2}{#3}}\xspace}       % con ()

\newcommand{\eleMM}[3]{\ensuremath{\MakeLowercase{#1}_{{#2}{#3}}}\xspace}       % con ()

\newcommand{\eleMMT}[3]{\ensuremath{\elemLLRR{\Mat{#1}^\T}{#2}{#3}}\xspace}   % con ()         
\newcommand{\eleMMM}[3]{\ensuremath{\elemLLLRRR{\Mat{#1}}{#2}{#3}}\xspace}    % con texto "elem"
\newcommand{\eleMMMT}[3]{\ensuremath{\elemLLLRRR{\Mat{#1}^\T}{#2}{#3}}\xspace}    % con texto "elem"

%SELECCIONA UN ELEMENTO DE UN VECTOR (por la derecha o la izquierda siginifica lo mismo en este caso)
\newcommand{\eleVL}[2]{\ensuremath{\elemL{\Vect{#1}}{#2}}\xspace}             % con subindices
\newcommand{\eleVR}[2]{\ensuremath{\elemR{\Vect{#1}}{#2}}\xspace}             % con subindices
\newcommand{\eleVLP}[2]{\ensuremath{\elemLP{\Vect{#1}}{#2}}\xspace}  % con subindices y paréntesis interior
\newcommand{\eleVRP}[2]{\ensuremath{\elemRP{\Vect{#1}}{#2}}\xspace}  % con subindices y paréntesis interior
\newcommand{\eleVLPE}[2]{\ensuremath{\elemLPE{\Vect{#1}}{#2}}\xspace}  % con subindices y paréntesis exterior
\newcommand{\eleVRPE}[2]{\ensuremath{\elemRPE{\Vect{#1}}{#2}}\xspace}  % con subindices y paréntesis exterior

\newcommand{\eleVUU}[2]{\ensuremath{\elemUU{\Vect{#1}}{#2}}\xspace}           % con ()      
\newcommand{\eleVUUU}[2]{\ensuremath{\elemUUU{\Vect{#1}}{#2}}\xspace}         % con ()      

%SELECCIÓNA de FILAS y COlUMNAS DE UNA MATRIZ PARA GENERAR UN VECTOR DE Rn
\newcommand{\VectC}[2][{}]   {\ensuremath{\elemR  {\Mat{#2}}{#1}}\xspace}   % con subindices
\newcommand{\VectF}[2][{}]   {\ensuremath{\elemL  {\Mat{#2}}{#1}}\xspace}   % con subindices
\newcommand{\VectCP}[2][{}]   {\ensuremath{\elemRP  {\Mat{#2}}{#1}}\xspace}   % con subindices y paréntesis
\newcommand{\VectFP}[2][{}]   {\ensuremath{\elemLP  {\Mat{#2}}{#1}}\xspace}   % con subindices y paréntesis
\newcommand{\VectCPE}[2][{}]   {\ensuremath{\elemRPE  {\Mat{#2}}{#1}}\xspace}   % con subindices y paréntesis exterior
\newcommand{\VectFPE}[2][{}]   {\ensuremath{\elemLPE  {\Mat{#2}}{#1}}\xspace}   % con subindices y paréntesis exterior

\newcommand{\VectCC}[2][{}]  {\ensuremath{\elemRR {\Mat{#2}}{#1}}\xspace}   % con ()
\newcommand{\VectFF}[2][{}]  {\ensuremath{\elemLL {\Mat{#2}}{#1}}\xspace}   % con ()
\newcommand{\VectCCC}[2][{}] {\ensuremath{\elemRRR{\Mat{#2}}{#1}}\xspace}   % con texto "col"
\newcommand{\VectFFF}[2][{}] {\ensuremath{\elemLLL{\Mat{#2}}{#1}}\xspace}   % con texto "fila" o "row"

%SELECCIÓNA de FILAS y COlUMNAS DE UNA MATRIZ TRANSPUESTA PARA GENERAR UN VECTOR DE Rn
\newcommand{\VectTC}[2][{}]  {\ensuremath{\elemR{\big(\MatT{#2}\big)}{#1}}\xspace}  % con subindices
\newcommand{\VectTF}[2][{}]  {\ensuremath{\elemL{\big(\MatT{#2}\big)}{#1}}\xspace}  % con subindices
\newcommand{\VectTCC}[2][{}] {\ensuremath{\elemRR{ \MatT{#2}}{#1}}\xspace}  % con ()
\newcommand{\VectTFF}[2][{}] {\ensuremath{\elemLL{ \MatT{#2}}{#1}}\xspace}  % con ()
\newcommand{\VectTCCC}[2][{}]{\ensuremath{\elemRRR{\MatT{#2}}{#1}}\xspace}  % con texto "col"
\newcommand{\VectTFFF}[2][{}]{\ensuremath{\elemLLL{\MatT{#2}}{#1}}\xspace}  % con texto "fila" o "row"

%% GENERA MATRIZ FILA CON UNA FILA DE UNA MATRIZ 
\newcommand{\VectFT}[2][{}]  {\ensuremath{\left[\VectF[#1]{#2}\vphantom{\big)}\right]^\T}\xspace}    % subindices
\newcommand{\VectFTB}[2][{}] {\ensuremath{\left[\VectF[#1]{#2}\vphantom{\Big)}\right]^\T}\xspace}   % subindices grande
\newcommand{\VectFFT}[2][{}] {\ensuremath{\left[\VectFF[#1]{#2}\vphantom{\big)}\right]^\T}\xspace}  % ()
\newcommand{\VectFFTB}[2][{}]{\ensuremath{\left[\VectFF[#1]{#2}\vphantom{\Big)}\right]^\T}\xspace} % () grande        
\newcommand{\VectFFFT}[2][{}] {\ensuremath{\left[\VectFFF[#1]{#2}\vphantom{\big)}\right]^\T}\xspace}  % texto
\newcommand{\VectFFFTB}[2][{}]{\ensuremath{\left[\VectFFF[#1]{#2}\vphantom{\Big)}\right]^\T}\xspace} % texto y grande        

%% GENERA MATRIZ FILA CON UNA COLUMNA DE UNA MATRIZ (con subindices)
\newcommand{\VectCT}[2][{}]  {\ensuremath{\left[\VectC[#1]{#2}\vphantom{\big)}\right]^\T}\xspace}    % subindices       
\newcommand{\VectCTB}[2][{}] {\ensuremath{\left[\VectC[#1]{#2}\vphantom{\Big)}\right]^\T}\xspace}   % subindices grande
\newcommand{\VectCCT}[2][{}] {\ensuremath{\left[\VectCC[#1]{#2}\vphantom{\big)}\right]^\T}\xspace}  % ()               
\newcommand{\VectCCTB}[2][{}]{\ensuremath{\left[\VectCC[#1]{#2}\vphantom{\Big)}\right]^\T}\xspace} % () grande        
\newcommand{\VectCCCT}[2][{}] {\ensuremath{\left[\VectCCC[#1]{#2}\vphantom{\big)}\right]^\T}\xspace}  % texto          
\newcommand{\VectCCCTB}[2][{}]{\ensuremath{\left[\VectCCC[#1]{#2}\vphantom{\Big)}\right]^\T}\xspace} % texto y grande        

%% GENERA MATRIZ COLUMNA CON UNA FILA DE UNA MATRIZ (con subindices)
\newcommand{\MVectF}[2][{}] {\ensuremath{\left[\VectF[#1]{#2}\vphantom{\big)}\right]}\xspace}       % subindices       
\newcommand{\MVectFB}[2][{}] {\ensuremath{\left[\VectF[#1]{#2}\vphantom{\Big)}\right]}\xspace}      % subindices grande
\newcommand{\MVectFF}[2][{}] {\ensuremath{\left[\VectFF[#1]{#2}\vphantom{\big)}\right]}\xspace}     % ()               
\newcommand{\MVectFFB}[2][{}] {\ensuremath{\left[\VectFF[#1]{#2}\vphantom{\Big)}\right]}\xspace}    % () grande        
\newcommand{\MVectFFF}[2][{}] {\ensuremath{\left[\VectFFF[#1]{#2}\vphantom{\big)}\right]}\xspace}     % texto
\newcommand{\MVectFFFB}[2][{}] {\ensuremath{\left[\VectFFF[#1]{#2}\vphantom{\Big)}\right]}\xspace}    % texto y grande        

%% GENERA MATRIZ COLUMNA CON UNA COLUMNA DE UNA MATRIZ (con subindices)
\newcommand{\MVectC}[2][{}] {\ensuremath{\left[\VectC[#1]{#2}\vphantom{\big)}\right]}\xspace}       % subindices       
\newcommand{\MVectCB}[2][{}] {\ensuremath{\left[\VectC[#1]{#2}\vphantom{\Big)}\right]}\xspace}      % subindices grande
\newcommand{\MVectCC}[2][{}] {\ensuremath{\left[\VectCC[#1]{#2}\vphantom{\big)}\right]}\xspace}     % ()               
\newcommand{\MVectCCB}[2][{}] {\ensuremath{\left[\VectCC[#1]{#2}\vphantom{\Big)}\right]}\xspace}    % () grande        
\newcommand{\MVectCCC}[2][{}] {\ensuremath{\left[\VectCCC[#1]{#2}\vphantom{\big)}\right]}\xspace}     % texto 
\newcommand{\MVectCCCB}[2][{}] {\ensuremath{\left[\VectCCC[#1]{#2}\vphantom{\Big)}\right]}\xspace}    % texto y grande        


%%%%%%%%%%%%%%%%%%%%%%%%%%%%%%%%%%%%%%%%%%%%%%%%%%%%%%%%%%%%%%%%%%%%%%%%%%%%%%%%%%%%%%%%%%%%%%%%%%%%%%%%%%%%%%%%%%%%%%%%%%%%
%%%%%%%%%%%%%%%%%%%%%%%%%%%%%%%%%%%%%%%%%%%%%%%%%%%%%%%%%%%%%%%%%%%%%%%%%%%%%%%%%%%%%%%%%%%%%%%%%%%%%%%%%%%%%%%%%%%%%%%%%%%%
%%%%%%%%%%%%%%%%%%%%%%%%%%%%%%%%%%%%%%%%%%%%%%%%%%%%%%%%%%%%%%%%%%%%%%%%%%%%%%%%%%%%%%%%%%%%%%%%%%%%%%%%%%%%%%%%%%%%%%%%%%%%

%% Matrices fila y columna (tipo vector)
%\newcommand{\VectF}[2][{}]{\ensuremath{\uvec{{\boldsymbol{\MakeLowercase{#2}}}_{#1}}_{\scriptscriptstyle\triangleright}}\xspace} 
%\newcommand{\VectC}[2][{}]{\ensuremath{\uvec{{\boldsymbol{\MakeLowercase{#2}}_{\scriptscriptstyle\blacktriangledown}}_{#1}}}\xspace} 
%\newcommand{\VectFT}[2][{}]{\ensuremath{\uvec{{\boldsymbol{\MakeLowercase{#2}}}^\T_{#1}}_{\scriptscriptstyle\triangleright}}\xspace} 
%\newcommand{\VectCT}[2][{}]{\ensuremath{\uvec{{\boldsymbol{\MakeLowercase{#2}}^\T_{\scriptscriptstyle\blacktriangledown}}_{#1}}}\xspace}

%% Matrices fila y columna (una mezcla matriz vector)
%\renewcommand{\VectF}[2][{}] {\ensuremath{\uvec{{\Mat{#2}}_{(#1)\triangleright}}}\xspace} 
%\renewcommand{\VectC}[2][{}] {\ensuremath{\uvec{{\Mat{#2}_{\scriptstyle\blacktriangledown\!}}_{(#1)}}}\xspace} 
%\renewcommand{\VectFT}[2][{}] {\ensuremath{(\uvec{{\Mat{#2}}_{(#1)\triangleright}})^\T}\xspace} 
%\renewcommand{\VectCT}[2][{}] {\ensuremath{(\uvec{{\Mat{#2}_{\scriptstyle\blacktriangledown\!}}_{(#1)}})^\T}\xspace} 

%% (A)_{jk}
%\renewcommand{\VectF}[2][{}] {\ensuremath{\left(\Mat{#2}\right)_{#1\triangleright}}\xspace} 
%\renewcommand{\VectC}[2][{}] {\ensuremath{\left(\Mat{#2}\right)_{{\scriptstyle\blacktriangledown}#1}}\xspace} 
%\renewcommand{\VectFT}[2][{}] {\ensuremath{\left(\Mat{#2}_{#1\triangleright}\right)^\T}\xspace} 
%\renewcommand{\VectCT}[2][{}] {\ensuremath{\left(\Mat{#2}_{{\scriptstyle\blacktriangledown}#1}\right)^\T}\xspace} 

%\renewcommand{\VectF}[2][{}] {\ensuremath{\Mat{#2}_{#1\triangleright}}\xspace} 
%\renewcommand{\VectC}[2][{}] {\ensuremath{\Mat{#2}_{{\scriptstyle\blacktriangledown}#1}}\xspace} 
%\renewcommand{\VectFT}[2][{}] {\ensuremath{\left[\Mat{#2}_{#1\triangleright}\vphantom{\big)}\right]^\T}\xspace} 
%\renewcommand{\VectCT}[2][{}] {\ensuremath{\left[\Mat{#2}_{{\scriptstyle\blacktriangledown}#1}\vphantom{\big)}\right]^\T}\xspace} 
%\newcommand{\MVectF}[2][{}] {\ensuremath{\left[\Mat{#2}_{#1\triangleright}\vphantom{\big)}\right]}\xspace} 
%\newcommand{\MVectC}[2][{}] {\ensuremath{\left[\Mat{#2}_{{\scriptstyle\blacktriangledown}#1}\vphantom{\big)}\right]}\xspace} 


%\newcommand{\VectFFT}[2][{}] {\ensuremath{\left[\Mat{#2}(#1,\triangleright)\vphantom{\big)}\right]^\T}\xspace} % ()
%\newcommand{\VectCCT}[2][{}] {\ensuremath{\left[\Mat{#2}({\scriptstyle\blacktriangledown},#1)\vphantom{\big)}\right]^\T}\xspace} 
%\newcommand{\MVectFF}[2][{}] {\ensuremath{\left[\Mat{#2}(#1,\triangleright)\vphantom{\big)}\right]}\xspace} 
%\newcommand{\MVectCC}[2][{}] {\ensuremath{\left[\Mat{#2}({\scriptstyle\blacktriangledown},#1)\vphantom{\big)}\right]}\xspace} 

%\renewcommand{\VectFT}[2][{}] {\ensuremath{\Mat{#2}_{#1\triangleright}^\T}\xspace} 
%\renewcommand{\VectCT}[2][{}] {\ensuremath{\Mat{#2}_{{\scriptstyle\blacktriangledown}#1}^\T}\xspace} 
%\renewcommand{\VectF}[2][{}] {\ensuremath{\left[\Mat{#2}_{#1\triangleright}\right]}\xspace} 
%\renewcommand{\VectC}[2][{}] {\ensuremath{\left[\Mat{#2}_{{\scriptstyle\blacktriangledown}#1}\right]}\xspace} 

%% A(g,h)
%\renewcommand{\VectF}[2][{}] {\ensuremath{\Mat{#2}(#1,\triangleright)}\xspace} 
%\renewcommand{\VectC}[2][{}] {\ensuremath{\Mat{#2}({\scriptstyle\blacktriangledown},#1)}\xspace} 
%\renewcommand{\VectFT}[2][{}] {\ensuremath{\left[\Mat{#2}(#1,\triangleright)\right]^\T}\xspace} 
%\renewcommand{\VectCT}[2][{}] {\ensuremath{\left[\Mat{#2}({\scriptstyle\blacktriangledown},#1)\right]^\T}\xspace} 

%\newcommand{\VectFF}[2][{}] {\ensuremath{\Mat{#2}(#1,\triangleright)}\xspace} 
%\newcommand{\VectCC}[2][{}] {\ensuremath{\Mat{#2}({\scriptstyle\blacktriangledown},#1)}\xspace}

%\renewcommand{\VectF}[2][{}] {\ensuremath{\Mat{#2}(#1,{\getItem})}\xspace} 
%\renewcommand{\VectC}[2][{}] {\ensuremath{\Mat{#2}({\getItem},#1)}\xspace} 



\def\first@element{1}
\renewcommand{\firstelement}[1]{\def\first@element{#1}}
% \renewcommand{\irvec}[2][n]{\Vect[\first@element]{#2},\ldots,\Vect[#1]{#2}}
% \renewcommand{\icvec}[2][n]{%
%   \begin{array}{c}
%     \Vect[\first@element]{#2}\\ \vdots\\ \Vect[#1]{#2}
%   \end{array}}
\newcommand{\irvecC}[2][n]{\VectC[\first@element]{#2},\ldots,\VectC[#1]{#2}}
\newcommand{\icvecF}[2][n]{%
  \begin{array}{c}
    \VectF[\first@element]{#2}\\ \vdots\\ \VectF[#1]{#2}
  \end{array}}
\newcommand{\icvecFF}[2][n]{% columna de filas puestas en fila separadas por ";"
  \begin{array}{c}
    \VectF[\first@element]{#2};\ldots;\VectF[#1]{#2}
  \end{array}}

% \newcounter{vecc@elemm}
% \newcommand{\rvecC}[3]{%
%   \ensuremath{%
%     \ifthenelse{#3 > #2}{%
%       \setcounter{vecc@elemm}{#2}
%       \whiledo{\value{vecc@elemm} < #3}%
%         {\VectC[\thevecc@elemm]{#1}, \stepcounter{vecc@elemm}}%
%       \VectC[#3]{#1}}{\VectC[#2]{#1}}}}
% \newcommand{\cvecF}[3]{%
%   \ifthenelse{#3 > #2}{%
%     \setcounter{vecc@elemm}{#2}
%     \begin{array}{c}
%       \whiledo{\value{vecc@elemm} < #3}%
%         {\VectF[\thevecc@elemm]{#1} \\ \stepcounter{vecc@elemm}}%
%       \VectF[#3]{#1}
%     \end{array}}{\VectF[#2]{#1}}}

%%%%%%%%%%%%%%%%%%%%%%%%%%%%%%%%%%%
% \newcommand{\Vect}[2][{}]{\ensuremath{{{\vect{#2}}_{#1}}}\xspace} 

% %\newcommand{\VectF}[2][{}]{\ensuremath{{\vect{{#2}}_{#1_\triangleright}}}\xspace} 

% \newcommand{\VectF}[2][{}]{\ensuremath{{\vect{{#2}}_{#1{\scriptscriptstyle\triangleright}}}}\xspace} 
% \newcommand{\VectC}[2][{}]{\ensuremath{{\vect{{#2}\!}_{{\scriptscriptstyle\blacktriangledown\!}#1}}}\xspace} 

% \newcommand{\VectCT}[2][{}]{\ensuremath{{\vect{{#2}}^\T_{{\scriptscriptstyle\blacktriangledown\!}#1}}}\xspace} 
% \newcommand{\VectFT}[2][{}]{\ensuremath{{\vect{{#2}}^\T_{#1{\scriptscriptstyle\triangleright}}}}\xspace} 
%%%%%%%%%%%%%%%%%%%%%%%%%%%%%%%%%%%%%%%%%%%%%%%%%%%%


%\RequirePackage{stmaryrd}
%\newcommand{\LIncplt}[2]{\ensuremath{{\left\llbracket #1 \right\rrbracket}_{\HLr{\xcancel{#2}}}}\xspace}
%\newcommand{\LIncplt}[2]{\ensuremath{{\left\langle #1 \right\rangle}_{\HLr{\xcancel{#2}}}}\xspace}
%\newcommand{\LIncplt}[2]{\ensuremath{{\left\lceil #1 \right\rceil}_{\HLr{\xcancel{#2}}}}\xspace}

\newcommand{\LIncplt}[2]{\ensuremath{{\left\llbracket #1 \right\rrbracket}_{\HLr{\xcancel{#2}}}}\xspace}

%\newcommand{\VectC}[2]{\ensuremath{{\vect{{#1}_{_\blacktriangledown}}_{#2}}}\xspace} 


%\newcommand{\VectCol}[2]{\ensuremath{{\vect{{#2}}_{{_\blacktriangledown}\!#1}}}\xspace} 
\newcommand{\MVC}[3][{}]{\ensuremath{\Mat{#2}\VectC[#1]{#3}}\xspace}%MatVectColumna
\newcommand{\VFM}[3][{}]{\ensuremath{\VectF[#1]{#2}\Mat{#3}}\xspace}%MatVectColumna

%% Vector con sus dimensiones 1x1
\newcommand{\dimensiones}[3]{\ensuremath{\mathop{#1}\limits_{ {\scriptscriptstyle #2\times #3} }}\xspace}
%\newcommand{\Vectdim}[3]{\ensuremath{\Mat{\MakeLowercase{#1}}\limits_{{[#2\times #3]}}\xspace}} 
\newcommand{\Vectdim}[3]{\Dim{\Vect{#1}}{#2}{#3}} 
%\newcommand{\Vectdim}[3]{\dimensiones{\vect{#1}}{#2}{#3}} 
\newcommand{\VectTdim}[3]{\Dim{\VectT{#1}}{#2}{#3}} 
%\newcommand{\VectTdim}[3]{\dimensiones{\vect{#1^\T}}{#2}{#3}} 
\newcommand{\Dim}[3]{\dimensiones{#1}{#2}{#3}} 

%\newcommand{\VectT}[1]{\ensuremath{{\Vect{#1}}^{\T}}\xspace} 
%\newcommand{\VectT}[1]{\ensuremath{{{\vect{#1}}^\T}}\xspace} 
%\newcommand{\VectT}[2][{}]{\ensuremath{{{\vect{#2}^{\T}}_{#1}}}\xspace} 

%\newcommand{\VectT}[2][{}]{\ensuremath{\Vect[#1]{#2}^{\T}}\xspace}


%% Matriz #1 de dimensiones #2 x #3
%\newcommand{\Matdim}[3]{\ensuremath{\Mat{#1}\limits_{{[#2\times #3]}}\xspace}}
\newcommand{\Matdim}[3]{\dimensiones{\mat{#1}}{#2}{#3}} 
%% Escalar con sus dimensiones 1x1
%\newcommand{\Escdim}[1]{\ensuremath{\mathop{#1}\limits_{[1\times1]}}\xspace}
\newcommand{\Escdim}[1]{\dimensiones{\mathop{#1}}{1}{1}} 
%% Matriz #1 traspuesta
%\newcommand{\MatT}[1]{\ensuremath{{\Mat{#1}\!}^{\T}\,}\xspace} 
\newcommand{\MatT}[2][{}]{\ensuremath{{{\mat{#2}}^{\T}_{#1}}}\xspace} 

%% Matriz #1 traspuesta de dimensiones #2 x #3
\newcommand{\MatTdim}[3]{\ensuremath{\mathop{\Matdim{#1^\T}{#2}{#3}}}\xspace}
%%%%%%%%%%%%%%%%%%%%%%%%%%%%%%%
%%% Operaciones sobre matrices
%\providecommand{\modulus}[1]{\ensuremath{\lvert#1\rvert}\xspace}  %MODULO
\providecommand{\norma}[1]{\ensuremath{\lVert#1\rVert}\xspace}  %MODULO
\newcommand{\modulus}[1]{\ensuremath{\left|#1\right|}\xspace}  %MODULO
\newcommand{\determinante}[1]{\modulus{#1}}  %Determinante
\newcommand{\menor}[3]{\ensuremath{M_{#1#2}(\Mat{#3})}\xspace}  %Determinante

\newcommand{\cofactor}[3]{\ensuremath{\cof_{#1#2}(\Mat{#3})}\xspace}  %Determinante


\DeclareMathOperator{\fila}{\eng{fila}{row}}
\DeclareMathOperator{\columna}{\eng{col}{column}}
\DeclareMathOperator{\cof}{cof}
\DeclareMathOperator{\adj}{Adj}

\newcommand{\EV}[1]{\ensuremath{\mathcal{#1}}}
\DeclareMathOperator{\EspacioNulo}{\EV{N}}
\DeclareMathOperator{\EspacioColumna}{\EV{C}}
\newcommand{\nulls}[1]{\ensuremath{\EspacioNulo\left({\Mat{#1}}\right)}\xspace}
\newcommand{\Nulls}[1]{\ensuremath{\EspacioNulo\left(#1\right)}\xspace}
\newcommand{\cols}[1]{\ensuremath{\EspacioColumna\left({\Mat{#1}}\right)}\xspace}

%\newcommand{\TEFI}[3]{\ensuremath{{\eng{F}{R}}_{\left[(\Vect[#1]{e}#3\Vect[#2]{e})\rightarrow\Vect[#1]{e}\right]}^\mathit{I}}}
\newcommand{\TEFI}[3]{\ensuremath{{\eng{F}{R}}_{
        \left[
           \substack{(\Vect[#1]{e}#3\Vect[#2]{e}) \\ \downarrow \\ \Vect[#1]{e}}
        \right]}^\mathit{I}}}
%\newcommand{\TEFII}[2]{\ensuremath{{\eng{F}{R}}_{\left[#2\Vect[#1]{e}\rightarrow\Vect[#1]{e}\right]}^\mathit{II}}}
\newcommand{\TEFII}[2]{\ensuremath{{\eng{F}{R}}_{
        \left[
           \substack{(#2\Vect[#1]{e}) \\ \downarrow \\ \Vect[#1]{e}}
        \right]}^\mathit{II}}}
%\newcommand{\TEIII}[2]{\ensuremath{{F}_{[#1\leftrightarrow #2]}^\mathit{III}}}
\newcommand{\TEFIII}[2]{\ensuremath{{\eng{F}{R}}_{\left[\Vect[#1]{e}\leftrightarrow\Vect[#2]{e}\right]}}}

\renewcommand{\TEFI}[3]{\ensuremath{{\eng{F}{R}}_{
        \left[
           \substack{(\bullet_{#1}{#3}\bullet_{#2}) \\ \downarrow \\ \bullet_{#2}}
        \right]}^\mathit{I}}}

%\renewcommand{\TEFII}[2]{\ensuremath{{\eng{F}{R}}_{
%        \left[
%          \substack{#2(\bullet_{#1}) \\ \downarrow \\ \bullet_{#1}}
%       \right]}^\mathit{II}}}
%
%\renewcommand{\TEFIII}[2]{\ensuremath{{\eng{F}{R}}_{
%        \left[
%          \substack{\bullet_{#1} \\ \updownarrow \\ \bullet_{#2}}
%       \right]}^\mathit{III}}}


%\newcommand{\TECI}[3]{\ensuremath{{C}_{\left[\Vect[#2]{e}\leftarrow(\Vect[#2]{e}#3\Vect[#1]{e})\right]}^\mathit{I}}}
%\newcommand{\TECII}[2]{\ensuremath{{C}_{\left[\Vect[#1]{e}\leftarrow#2\Vect[#1]{e}\right]}^\mathit{II}}}
\newcommand{\TECI}[3]{\ensuremath{{C}_{\left[(\Vect[#2]{e}#3\Vect[#1]{e})\rightarrow\Vect[#2]{e}\right]}^\mathit{I}}}
\newcommand{\TECII}[2]{\ensuremath{{C}_{\left[(#2\Vect[#1]{e})\rightarrow\Vect[#1]{e}\right]}^\mathit{II}}}
\newcommand{\TECIII}[2]{\ensuremath{{C}_{\left[\Vect[#1]{e}\leftrightarrow\Vect[#2]{e}\right]}}}

\renewcommand{\TECI}[3]{\ensuremath{{C}_{\left[\su{#1}{#2}{#3}\right]}^\mathit{I}}}
\renewcommand{\TECII}[2]{\ensuremath{{C}_{\left[\pr{#1}{#2}\right]}^\mathit{II}}}
\renewcommand{\TEFI}[3]{\ensuremath{{F}_{\left[\su{#1}{#2}{#3}\right]}^\mathit{I}}}
\renewcommand{\TEFII}[2]{\ensuremath{{F}_{\left[\pr{#1}{#2}\right]}^\mathit{II}}}

% \newcommand{\TEFI}[3]{\ensuremath{{F}_{\left[\VectF[#1]{e}#3\VectF[#2]{e}\rightarrow\VectF[#1]{e}\right]}^\mathit{I}}}
% \newcommand{\TEFII}[2]{\ensuremath{{F}_{\left[#2\VectF[#1]{e}\rightarrow\VectF[#1]{e}\right]}^\mathit{II}}}
% %\newcommand{\TEIII}[2]{\ensuremath{{F}_{[#1\leftrightarrow #2]}^\mathit{III}}}
% \newcommand{\TEFIII}[2]{\ensuremath{{F}_{\left[\VectF[#1]{e}\leftrightarrow\VectF[#2]{e}\right]}}}

% \newcommand{\TEFI}[3]{\ensuremath{{F}_{[\Vect{#1} #3 \Vect{#2}]}^\mathit{I}}}
% \newcommand{\TEFII}[2]{\ensuremath{{F}_{[#2 \Vect{#1}]}^\mathit{II}}}
% %\newcommand{\TEIII}[2]{\ensuremath{{F}_{[#1\leftrightarrow #2]}^\mathit{III}}}
% \newcommand{\TEFIII}[2]{\ensuremath{{F}_{[\Vect{#1}\leftrightarrow \Vect{#2}]}}}


% \newcommand{\TECI}[3]{\ensuremath{{C}_{[\Vect{#2} #3 \Vect{#1}]}^\mathit{I}}}
% \newcommand{\TECII}[2]{\ensuremath{{C}_{[#2 \Vect{#1}]}^\mathit{II}}}
% \newcommand{\TECIII}[2]{\ensuremath{{C}_{[\Vect{#1}\leftrightarrow \Vect{#2}]}}}

%\DeclareMathOperator{\traza}{traza}
%\newcommand{\Traza}[1]{\ensuremath{\uniforme\left(#1,\ #2\right)}\xspace}


%%%%%%%%%%%%%%%%%%%%%%%%%%%%%%%
%%% PROBABILIDAD
%%%%%%%%%%%%%%%%%%%%%%%%%%%%%%%

%% colores
%\newcommand{\colestim}{blue}
%\newcommand{\colorVA}{blue}
%\newcommand{\coldeterm}{black}
\newcommand{\colestim}{magenta}
\newcommand{\colorVA}{magenta}
\definecolor{MAGENTA}{rgb}{0.9,0,0.4} % magenta más o menos
\definecolor{BLUE}{rgb}{0,0,1} % magenta más o menos

%%%%%%%%% http://latexcolor.com/ %%%%%%%%%%
\definecolor{coloraleatorio}{rgb}{0.6,0.1,.3} % granate
%\definecolor{coloraleatorio}{rgb}{0.0, 0.5, 0.0} % ao(english)
%\definecolor{coloraleatorio}{rgb}{0.65, 0.04, 0.37} % jazzberryjam
%\definecolor{coloraleatorio}{rgb}{0.11, 0.35, 0.02} % lincolngreen
%\definecolor{coloraleatorio}{rgb}{0.29, 0.33, 0.13}  % armygreen se ve poco
%\definecolor{coloraleatorio}{rgb}{0.79, 0.0, 0.09} % harvardcrimson

%\definecolor{coloraleatorio}{rgb}{0.0, 0.26, 0.15} %britishracinggreen
%\definecolor{coloraleatorio}{rgb}{0.37, 0.62, 0.63} %cadetblue
%\definecolor{coloraleatorio}{rgb}{0.2, 0.2, 0.6} % blue(pigment)
%\definecolor{coloraleatorio}{rgb}{0.12, 0.3, 0.17} %calpolypomonagreen
%\definecolor{coloraleatorio}{rgb}{0.09, 0.45, 0.27} % darkspringgreen
%\definecolor{coloraleatorio}{rgb}{0.0, 0.27, 0.13} % forestgreen(traditional)
%\definecolor{coloraleatorio}{rgb}{0.13, 0.55, 0.13} % forestgreen(web)
\renewcommand{\colestim}{coloraleatorio}
\renewcommand{\colorVA}{coloraleatorio}

%% espacio de probabilidad
\newcommand{\ExpAleat}{\ensuremath{\mathcal{E}}\xspace}
\newcommand{\EspMuestral}{\ensuremath{S}\xspace}
\newcommand{\EspSucesos}{\ensuremath{\mathfrak{B}}\xspace}
\newcommand{\probspace}{\ensuremath{(\EspMuestral,\EspSucesos,\Prb{\cdot})}\xspace}
\newcommand{\Ensayo}{\ensuremath{\mathcal{A}}\xspace}

%%% COND !!!!!!!!!
\newcommand{\cOND}[2]{\ensuremath{\left.{#1}\,\right|\left.#2\!\right.}\xspace}
\newcommand{\coND}[2]{\ensuremath{\left.\left(\!\cOND{#1}{#2}\!\right.\right)}\xspace}
\newcommand{\Cnd}[2]{\funcion{#1}{|{\scriptscriptstyle#2}}}
\newcommand{\CnD}[2]{\COND{#1}{#2}}

\newcommand{\PVEstmdCnd}[2]{\ensuremath{\left(\VEstmdCnd{#1}{#2}\right)}\xspace}
\newcommand{\PEstmdCnd}[2]{\ensuremath{\left(\VEstmdCnd{#1}{#2}\right)}\xspace}


%% leyes de probabilidad
\newcommand{\probab}{{P}}
\newcommand{\Prb}[1]{\ensuremath{\probab\left(#1\right)}\xspace}
\newcommand{\PrbMg}[2]{
  \ensuremath{{\probab}_{\VA{#1}}\left(\VA{#1}=\MakeLowercase{#1}_{#2}\right)}\xspace}
\newcommand{\Prbcond}[2]{\Prb{\cOND{#1}{#2}}}

\newcommand{\PrbH}[2]{\ensuremath{{\probab}_{\tiny #2}\left(#1\right)}\xspace}
%\newcommand{\PrbH}[2]{\ensuremath{{\probab}\left(#1\|#2\right)}\xspace}
%\newcommand{\PrbH}[2]{\ensuremath{{\probab}\left(#1\right)_{\tiny #2}}\xspace}

\newcommand{\funDist}{{F}}
\newcommand{\funDens}{{\MakeLowercase{\funDist}}}

%% variable aleatoria
%\newcommand{\VAn}[2]{\ensuremath{{\color{\colorVA}{\MakeUppercase{#1}_{#2}}}}\xspace}
\newcommand{\VAn}[2]{\ensuremath{{\color{\colorVA}{\MakeUppercase{#1}_{#2}}}}\xspace}
%\newcommand{\VAi}[2][{}]{\ensuremath{{\MakeUppercase{\color{\colorVA}#2}_{{\scriptscriptstyle\!}#1}}}\xspace}

\newcommand{\VAi}[2][{}]{\ensuremath{{\color{\colorVA}{\MakeUppercase{#2}_{{\scriptscriptstyle\!}#1}}}}\xspace}

%\newcommand{\VAi}[2][{}]{\ensuremath{{\color{\colorVA} \MakeUppercase{#2}_{\!#1}}}\xspace}
%\newcommand{\VA}[1]{{\VAn{#1}{}}}
\newcommand{\VA}[1]{{\VAi{#1}{\,}}}

\newcommand{\soporte}[2]{\ensuremath{\mathbb{#1}_{\scriptscriptstyle\color{\colorVA}{#2}}}\xspace}
\newcommand{\soporten}[3]{\ensuremath{\mathbb{#1}^{#3}_{\scriptscriptstyle\color{\colorVA}{#2}}}\xspace}


%% NOTACIÓN GENÉRICA PARA LEY DE PROBABILIDAD Y OPERADORES
\newcommand{\Leyx}[3]{\ensuremath{{#3}_{\!\;\scriptscriptstyle{\color{\colorVA}{#1}} }\!\!\: \left( #2 \right)}\xspace}
\newcommand{\Ley}[2]{\Leyx{}{#1}{#2}}
\newcommand{\leycond}[3]{\Ley{\cOND{#1}{#2}}{#3}}
\newcommand{\LeyCond}[3]{{\color{\colorVA}{\leycond{\!#1}{\!#2}{#3}}}}
\newcommand{\LeyconD}[4]{\LeycoND{#1}{#1}{#2}{#3}{#4}}
\newcommand{\LeyconDc}[5]{\LeycoND{#1}{#2}{#3}{#4}{#5}}
\newcommand{\Leycond}[3]{\LeyconD{#1}{#2}{#2}{#3}}
\newcommand{\LeYCx}[5]{\ensuremath{{#5}\left({#1}{#2}{#3}{#4}\right)}\xspace}
\newcommand{\LEYCX}[5]{\LeYcX{#1}{#3}{#2}{#4}{#5}}
\newcommand{\LeYc}[3]{\LeYcx{#1}{#2}{#3}}
\newcommand{\LeYC}[5]{\LEYCX{#1}{#3}{#2}{#4}{#5}}
%Condicion en minisculas
%\newcommand{\LeYcoNDc}[5]{\LeycoND{#1}{\VA{#1}=\MakeLowercase{#2}}{#3}{#4}{#5}}


\newcommand{\COND}[2]{\ensuremath{\left.{#1}\right|\!\!\left.#2\!\right.}\xspace}
%\newcommand{\COND}[2]{\ensuremath{{#1}}\xspace}
%\newcommand{\Parentesis}[1]{\ensuremath{\left(\!{#1}\right)}\xspace}
\newcommand{\Parentesis}[1]{\ensuremath{{#1}}\xspace}

%%%%%%%%%%%%%%%%%%%%%%%%%%%%% NOTACIÓN NOVALES %%%%%%%%%%
%% |X=x
%\newcommand{\LeYcoNDc}[5]{\LeycoND{#1}{\VA{#1}={#2}}{#3}{#4}{#5}}
%\newcommand{\LeYcX}[5]{\LeYCx{\VA{#1}}{=#2}{\,,\VA{#3}}{=#4}{#5}}
%\newcommand{\LeYcx}[3]{\LeYCx{\,\VA{#1}}{=#2}{}{}{#3}}
%\newcommand{\LeycoND}[5]{\leycond{#2}{\scriptstyle{\VA{#3}={#4}}}{#5}}
%% Condicion en minisculas
%\newcommand{\LeycoND}[5]{\leycond{#2}{\scriptstyle{\VA{#3}=\MakeLowercase{#4}}}{#5}}

%%%%%%%%%%%%%%%%%%% NOTACIÓN A LA HAYASHI (HYPERCOMPACTA) %%%%%%%%%%
% |x
\newcommand{\LeYcoNDc}[5]{\LeycoND{#1}{{#2}}{#3}{#4}{#5}}
\newcommand{\LeYcX}[5]{\LeYCx{\VA{#1}}{=#2}{\,,\VA{#3}}{=#4}{#5}}
\newcommand{\LeYcx}[3]{\LeYCx{\,\VA{#1}}{=#2}{}{}{#3}}
\newcommand{\LeycoND}[5]{\leycond{#2}{{#4}}{#5}}
\renewcommand{\Leyx}[3]{\ensuremath{{#3}\!\!\:\left( #2 \right)}\xspace}
%% Condicion en minisculas
%\newcommand{\LeycoND}[5]{\leycond{#2}{\scriptstyle{\VA{#3}=\MakeLowercase{#4}}}{#5}}

% %%%%%%%%%%%%%%%%%%%%%%%%%%%%% NOTACIÓN ALTERNATIVA (extensa) %%%%%%%%%%
% \newcommand{\LeYcoNDc}[5]{\LeycoND{#1}{\MakeLowercase{#2}}{#3}{#4}{#5}}
% \newcommand{\LeYcX}[5]{\Leyx{\!#1\!#3}{#2,#4}{#5}}
% \newcommand{\LeYcx}[3]{\Leyx{#1}{#2}{#3}}
% \newcommand{\LeycoND}[5]{\Leyx{{#1\!|\!#3}}{\cOND{#2}{{#4}}}{#5}}
% \newcommand{\LeyCoND}[3]{{\leyCond{#1}{#2}{#3}}}
%% Condicion en minisculas
%\newcommand{\LeycoND}[5]{\Leyx{#1\!|\!#3}{\cOND{#2}{\MakeLowercase{#4}}}{#5}}

%%%%%%%%%%%%%%%%%%%%%%%%%%%%% F. DISTRIB %%%%%%%%%%
\newcommand{\FdPx}[2]   {\Leyx{\!#1}{#2}{\funDist}}
\newcommand{\FdP}[2]    {\Leyx{\!#1}{#2}{\funDist}}
\newcommand{\Fdp}[1]    {\Leyx{\!#1}{\MakeLowercase{#1}}{\funDist}}
\newcommand{\FdpC}[4]   {\Leyx{\!#1\!#3}{#2,#4}{\funDist}}
\newcommand{\Fdpc}[2]   {\Leyx{\!#1\!#2}{\MakeLowercase{#1,#2}}{\funDist}}
\newcommand{\FdpcoND}[4]{\LeycoND{\!\!\;#1}{#2}{#3}{#4}{\funDist}}
\newcommand{\FdpconD}[3]{\LeycoND{\!\!\;\VA{#1}}{\MakeLowercase{#1}}{\VA{#2}}{#3}{\funDist}}
\newcommand{\Fdpcond}[2]{\fdpconD{\!\!\;#1}{\VA{#2}}{\MakeLowercase{#2}}}
\newcommand{\Fdppar}[1]{\FdP{#1}{\MakeLowercase{#1};\Mat{\theta}}}% func. densid. paramet
\newcommand{\FdpCpar}[4]{\FdP{#1#3}{#2,#4;{\pmb{\theta}}}}
\newcommand{\Fdpcpar}[2]{\FdpCpar{#1}{\MakeLowercase{#1}}{#2}{\MakeLowercase{#2}}}

%%%%%%%%%%%%%%%%%%%%%%%%%%%%% F. DENSIDAD %%%%%%%%%%
\newcommand{\fdPx}[2]   {\Leyx{\!\!\!\;#1}{#2}{\funDens}}
\newcommand{\fdP}[2]    {\Leyx{\!\!\!\;\MakeUppercase{#1}}{#2}{\funDens}}
\newcommand{\fdp}[1]    {\Leyx{\!\!\!\;\MakeUppercase{#1}}{\MakeLowercase{#1}}{\funDens}}
\newcommand{\fdpC}[4]   {\Leyx{\!\!\!\;#1\!#3}{#2,#4}{\funDens}}
\newcommand{\fdpc}[2]   {\Leyx{\!\!\!\;\MakeUppercase{#1\!#2}}{\MakeLowercase{#1,#2}}{\funDens}}
\newcommand{\fdpcoND}[4]{\LeycoND{\!\!\;#1}{#2}{#3}{#4}{\funDens}}
\newcommand{\fdpconD}[3]{\LeycoND{\!\!\;\VA{#1}}{\MakeLowercase{#1}}{\VA{#2}}{#3}{\funDens}}
\newcommand{\fdpcond}[2]{\fdpconD{\!\!\;#1}{\VA{#2}}{\MakeLowercase{#2}}}
\newcommand{\fdppar}[1]{\fdP{#1}{\MakeLowercase{#1};\Mat{\theta}}}%func. densid. paramet
\newcommand{\fdpCpar}[4]{\fdP{\!\!\!\;#1\!#3}{#2,#4;{\pmb{\theta}}}}
\newcommand{\fdpcpar}[2]{\fdpCpar{#1}{\MakeLowercase{#1}}{#2}{\MakeLowercase{#2}}}

\newcommand{\fv}[1]{\fdP{#1}{\cOND{\Vect{\theta}}{#1}}}%func. de verosimilitud


%%%%%%%%%%%%%%%%%%%%%%%%%%%%% F. CUANTÍA %%%%%%%%%%
\newcommand{\fcuanx}[2]   {\LeYc{\!\!\;\MakeUppercase{#1}}{#2}{\probab}}
\newcommand{\fcuaN}[2]    {\LeYc{\!\!\;\MakeUppercase{#1}}{#2}{\probab}}
\newcommand{\fcuan}[1]    {\LeYc{\!\!\;\MakeUppercase{#1}}{\MakeLowercase{#1}}{\probab}}
\newcommand{\fcuanC}[4]   {\LeYC{\!\!\;#1}{#2}{#3}{#4}{\probab}}
\newcommand{\fcuanc}[2]   {\LeYC{\!\!\;\MakeUppercase{#1}}{\MakeLowercase{#1}}{\MakeUppercase{#2}}{\MakeLowercase{#2}}{\probab}}
\newcommand{\fcuancoND}[4]{\LeYcoNDc{\!\!\;\VA{#1}}{#2}{#3}{#4}{\probab}}
\newcommand{\fcuanconD}[3]{\LeYcoNDc{\!\!\;\VA{#1}}{#1}{\VA{#2}}{#3}{\probab}}
\newcommand{\fcuanpar}[1] {\fcuaN{#1}{\MakeLowercase{#1};\Mat{\theta}}}
\newcommand{\fcuancond}[2]{\fcuanconD{#1}{\VA{#2}}{\MakeLowercase{#2}}     }


%%%%%%%%%%%%%%%%%%%%%%%%%%%% FAMILIA DE DISTRIBUCIONES
\DeclareMathOperator{\uniforme}{Uniforme}
\newcommand{\Uniforme}[2]{\ensuremath{\uniforme\left(#1,\ #2\right)}\xspace}
\DeclareMathOperator{\binomial}{Binomial}
\newcommand{\Binomial}[2]{\ensuremath{\binomial\left(#1,\ #2\right)}\xspace}
\DeclareMathOperator{\bernulli}{Bernulli}
\newcommand{\Bernulli}[1]{\ensuremath{\bernulli\left(#1\right)}\xspace}
\DeclareMathOperator{\poisson}{Poisson}
\newcommand{\Poisson}[1]{\ensuremath{\poisson\left(#1\right)}\xspace}
%\DeclareMathOperator{\normal}{Normal}
\DeclareMathOperator{\normal}{N\/}%{\it N\/}
\newcommand{\Normal}[2]{\ensuremath{\normal\left(#1\,,\,#2\right)}\xspace}
%\DeclareMathOperator{\tstudent}{\emph{t}-Student}
\DeclareMathOperator{\tstudent}{\it t\/}
\newcommand{\TStudent}[1]{\ensuremath{\tstudent_{\left\{#1\right\}}}\xspace}
\DeclareMathOperator{\fsnedecor}{\it F\/}
\newcommand{\FSnedecor}[2]{\ensuremath{\fsnedecor_{\left\{#1,#2\right\}}}\xspace}
%\DeclareMathOperator{\chicuadrado}{\ensuremath{\chi^2}\/}
\newcommand{\ChiCuadrado}[1]{\ensuremath{{\chi^2}_{\left\{ #1\right\}}\xspace}}


%% vbles #1 idep. e idénticamente distribuidas N(#2,#3) 
\newcommand{\iidNmv}[3]{\ensuremath{{#1}\sim\textrm{iid. }\Normal{#2}{#3}}\xspace}
%% vbles #1 idep. e idénticamente distribuidas N(\mu,sigma^2) 
\newcommand{\iidN}[1]{\iidNmv{#1}{\mu}{\sigma^2}}


%% Límite en probabilidad
\newcommand{\plim}{\ensuremath{\textrm{\ plim }}\xspace}
%% Varianza asintótica
\newcommand{\VarA}[1]{\ensuremath{\textrm{Var A}\left[{#1}\right]}\xspace}
%% distribución asintótica
\newcommand{\asim}{\mathop{\sim}\limits^a}
\newcommand{\simHnula}{\mathop{\sim}\limits_{\Hnula}}
\newcommand{\asimHnula}{\mathop{\sim}\limits_{\Hnula}^a}
% convergencia en distribución
\DeclareMathOperator{\distribucion}{d}
\newcommand{\ConvergenciaDist}{\ensuremath{\mathop{\rightarrow}\limits^{\distribucion}}\xspace}


%% MOMENTOS 
\DeclareMathOperator{\esperanza}{E}
\DeclareMathOperator{\mediana}{Me}
\DeclareMathOperator{\varianza}{Var}
\DeclareMathOperator{\covarianza}{Cov}
\DeclareMathOperator{\correlacion}{Corr}
\DeclareMathOperator{\DesviacionTipica}{Dt}

%% MOMENTOS  MUESTRALES
\newcommand{\Media}{m}
\newcommand{\Correlacion}{r}
\newcommand{\Covarianza}{s}
\newcommand{\Varianza}{\Covarianza}

%%%%%%%%%% ESPERANZA
\newcommand{\Ex}[2]{\Leyx{\VA{#1}}{#2}{\esperanza}}
\newcommand{\E}[1]{\Ley{#1}{\esperanza}}
\newcommand{\econd}[2]{\leycond{#1}{#2}{\esperanza}}
\newcommand{\ECond}[2]{\LeyCond{#1}{#2}{\esperanza}}
\newcommand{\EcoND}[4]{\LeycoND{#1}{#2}{#3}{#4}{\esperanza}}
\newcommand{\EconD}[3]{\LeycoND{#1}{#1}{\VA{#2}}{#3}{\esperanza}}
\newcommand{\Econd}[2]{\LeyconD{#1}{\VA{#2}}{\MakeLowercase{#2}}{\esperanza}}

\newcommand{\estimecond}[2]{\leycond{#1}{#2}{\widehat{\esperanza}}}

%%%%%%%%%% VARIANZA
\newcommand{\Varx}[2]{\Leyx{\VA{#1}}{#2}{\varianza}}
\newcommand{\Var}[1]{\Ley{#1}{\varianza}}
\newcommand{\varcond}[2]{\leycond{#1}{#2}{\varianza}}
\newcommand{\VarCond}[2]{\LeyCond{#1}{#2}{\varianza}}
\newcommand{\VarcoND}[4]{\LeycoND{#1}{#2}{#3}{#4}{\varianza}}
\newcommand{\VarconD}[3]{\LeycoND{#1}{#1}{\VA{#2}}{#3}{\varianza}}
\newcommand{\Varcond}[2]{\LeyconD{#1}{\VA{#2}}{\MakeLowercase{#2}}{\varianza}}
%%%%%%%%%% DESVIACIÓN TÍPICA
\newcommand{\Dtx}[2]{\Leyx{\VA{#1}}{#2}{\DesviacionTipica}}
\newcommand{\Dt}[1]{\Ley{#1}{\DesviacionTipica}}
\newcommand{\dtcond}[2]{\leycond{#1}{#2}{\DesviacionTipica}}
\newcommand{\DtCond}[2]{\LeyCond{#1}{#2}{\DesviacionTipica}}
\newcommand{\DtcoND}[4]{\LeycoND{#1}{#2}{#3}{#4}{\DesviacionTipica}}
\newcommand{\DtconD}[3]{\LeycoND{#1}{#1}{\VA{#2}}{#3}{\DesviacionTipica}}
\newcommand{\Dtcond}[2]{\LeyconD{#1}{\VA{#2}}{#2}{\DesviacionTipica}}

\newcommand{\MVAR}[1]{\ensuremath{{\Mat{\Sigma}}_{#1}}\xspace}
\newcommand{\MVARM}[1]{\ensuremath{\Mat{{\MakeUppercase\Varianza}^2_{\scriptscriptstyle{#1}}}}\xspace}
% \newcommand{\VAR}[1]{\ensuremath{\sigma^2_{\!\scriptscriptstyle\VA{#1}}}\xspace}
% \newcommand{\DT}[1]{\ensuremath{\sigma_{\!\scriptscriptstyle\VA{#1}}}\xspace}
% \newcommand{\ESP}[1]{\ensuremath{\mu_{\!\scriptscriptstyle\VA{#1}}}\xspace}
\newcommand{\VAR}[1]{\ensuremath{\sigma^2_{\!\scriptscriptstyle{#1}}}\xspace}
\newcommand{\DT}[1]{\ensuremath{\sigma_{\!\scriptscriptstyle{#1}}}\xspace}
\newcommand{\ESP}[1]{\ensuremath{\mu_{\!\scriptscriptstyle{#1}}}\xspace}


%%%%%%%%%% COVARIANZAS
\newcommand{\Covx}[3]{\Leyx{#1,#2}{#2}{\covarianza}}
\newcommand{\Cov}[2]{\Ley{#1,#2}{\covarianza}}
\newcommand{\CovconD}[4]{\LeycoND{#1#2}{#1,#2}{\VA{#3}}{#4}{\covarianza}}
\newcommand{\Covcond}[3]{\LeyconDc{#1#2}{#1,#2}{\VA{#3}}{#3}{\covarianza}}
\newcommand{\CovcoND}[6]{\LeycoND{#1#3}{#2,#4}{#5}{#6}{\covarianza}}
%\newcommand{\CovcoND}[6]{\LeycoND{#1}{#2,#4}{#3}{#5}{\covarianza}}

%\newcommand{\COV}[2]{\ensuremath{\sigma_{\!\scriptscriptstyle\VA{#1\!#2}}}\xspace}
\newcommand{\COV}[2]{\ensuremath{\sigma_{\!\scriptscriptstyle{#1\!#2}}}\xspace}
\newcommand{\VCOV}[2]{\ensuremath{\Vect{\sigma_{\!\scriptscriptstyle{#1\!#2}}}}\xspace}
\newcommand{\VCOVM}[2]{\ensuremath{\Vect{{\MakeUppercase\Varianza}_{\scriptscriptstyle{#1#2}}}}\xspace}
%\newcommand{\VTCOV}[2]{\ensuremath{\Vect{\sigma}^\T_{\!\scriptscriptstyle{#1\!#2}}}\xspace}

\newcommand{\Corr}[2]{\ensuremath{\correlacion\left(#1,#2\right)}\xspace}
%\newcommand{\CORR}[2]{\ensuremath{\rho_{\!\scriptscriptstyle\VA{#1\!#2}}}\xspace}
\newcommand{\CORR}[2]{\ensuremath{\rho_{\!\scriptscriptstyle{#1#2}}}\xspace}
\newcommand{\CORRM}[2]{\ensuremath{\Correlacion_{\!\scriptscriptstyle{#1#2}}}\xspace}

% correlación parcial muestral
%\newcommand{\CORRPM}[3]{\ensuremath{r_{\!\scriptscriptstyle{\ORT{#1}{#3}\!\ORT{#2}{#3}}}}\xspace}
\newcommand{\CORRPM}[3]{\ensuremath{\Correlacion_{\!\scriptscriptstyle{\ORT{(#1,#2)}{#3}}}}\xspace}

%%%%%%%%%%%%%%%%%% ESTADISTICA

%\newcommand{\Estad}[1]{\ensuremath{{\color{\colestim}{#1}}}\xspace}

%% estadístico #1 función de #2
%\newcommand{\Estd}[2]{\funcion{#1}{#2}}  % probablemente innecesario
\newcommand{\Estd}[2]{\ensuremath{{#1}_{#2}}}
\newcommand{\EstD}[2][{}]{\ensuremath{{#2}_{#1}}}

%\newcommand{\estadistico}[2]{\Estd{{\color{\colestim}{#1}}}{\left(#2\right)}}
%\newcommand{\estadistico}[1]{\ensuremath{\widehat{\mbox{\color{\colestim}{\Estd{#1}{}}}}}\xspace}
\newcommand{\estadistico}[2]{\ensuremath{{{\color{\colestim}{{#1}{(#2)}}}}}\xspace}

%% Estimador #1 que es función de #2     % probablemente innecesario
\newcommand{\Estm}[2]{\Estd{\color{\colestim}{#1}}{#2}} 

\newcommand{\estm}[2]{\mathop{\widehat{\EstD{#1}{#2}}}} 

%% Estimador #1 
\newcommand{\Estmd}[1]{\Estm{\widehat{#1}}{}}
%% Estimación de #1 
%\newcommand{\Estmc}[1]{\Estd{\widehat{#1}}{}} 
%\newcommand{\Estmc}[2][{}]{\EstD{\widehat{#1}{#2}}} 
%\newcommand{\Estmc}[1]{\estm{\widehat{#1}}{}}
\newcommand{\Estmc}[1]{\estm{#1}{}}

%% Estimador #1 condicionado #2
\newcommand{\Estmdcond}[2]{\widehat{\Estm{#1}{|#2}}}


%\newcommand{\VEstmdCnd}[2]{\COND{\color{\colestim}{\Vect{\widehat{#1}}}}{#2}}
%\newcommand{\EstmdCnd}[2]{\COND{\color{\colestim}{\widehat{#1}}}{#2}}
\newcommand{\EstmdCnd}[2]{\Estm{\widehat{#1}}{|{\scriptscriptstyle#2}}}
%\newcommand{\VEstmdCnd}[2]{\Estm{\Vect{\widehat{#1}}}{|#2}}
\newcommand{\VEstmdCnd}[2]{\Vect{\Estm{\widehat{#1}}{|{\scriptscriptstyle#2}}}}


\newcommand{\VEstmdCndP}[2]{\Parentesis{{\VEstmdCnd{#1}{#2}}}}
\newcommand{\EstmdCndP}[2]{\Parentesis{{\EstmdCnd{#1}{#2}}}}
%% PARA SUSTITUIR 
%\cOND{\Vect{\Estmd{\beta}}}{\Mat{X}} 
%% POR
%\VEstmdCnd{\beta}{\Mat{X}
%\EstmdVar{\VEstmdCnd{\beta}{\Mat{X}}}



%% Media aritmética o media muestral
\newcommand{\media}[1]{\ensuremath{\overline{#1}}\xspace}
%% varianza muestral
\newcommand{\MediaM}[1]{\ensuremath{\Media_{#1}}\xspace}
%% varianza muestral
\newcommand{\VarM}[1]{\ensuremath{\Varianza^2_{#1}}\xspace}
%% varianza muestral
\newcommand{\DTM}[1]{\ensuremath{\Varianza_{#1}}\xspace}
%% Covarianza muestral
\newcommand{\CovM}[2]{\ensuremath{\Covarianza_{#1#2}}\xspace}
%% Correlación muestral
\newcommand{\CorM}[2]{\ensuremath{\Correlacion_{#1#2}}\xspace}
%% cuasi-varianza muestral
\newcommand{\CVarM}[1]{\ensuremath{\mathfrak{s}^2_{#1}}\xspace}

\newcommand{\CVarMd}[1]{\Estmd{\CVarM{#1}}}


%% Estimador de la matriz de covarianzas de #1 que depende de #2
\newcommand{\EstmMedia}[2]{\ensuremath{\Estm{\media{#1}}{#2}}\xspace} 
%% Estimador de la matriz de covarianzas de #1 que depende de #2
\newcommand{\EstmVAR}[2]{\ensuremath{\Estm{\VAR{#1}}{#2}}\xspace} 
%% Estimador de la desviación típica (matriz) de #1 que depende de #2
\newcommand{\EstmDT}[2]{\ensuremath{\EstmVAR{#1}{#2}^{\frac{1}{2}}}\xspace}


\newcommand{\EstmdDt}[1]{\ensuremath{\Ley{#1}{\color{\colestim}{\widehat{\DesviacionTipica}}}}\xspace}
\newcommand{\EstmdVar}[1]{\ensuremath{\Ley{#1}{\color{\colestim}{\widehat{\varianza}}}}\xspace}
\newcommand{\EstmdCov}[2]{\ensuremath{\Ley{#1,#2}{\color{\colestim}{\widehat{\covarianza}}}}\xspace}

\newcommand{\EstmcDt}[1]{\ensuremath{\Ley{#1}{\widehat{\DesviacionTipica}}}\xspace}
\newcommand{\EstmcVar}[1]{\ensuremath{\Ley{#1}{\widehat{\varianza}}}\xspace}
\newcommand{\EstmcCov}[2]{\ensuremath{\Ley{#1,#2}{\widehat{\covarianza}}}\xspace}

%\newcommand{\EstmcDtcond}[1]{\ensuremath{\Ley{#1}{\widehat{\DesviacionTipica}}}\xspace}
\newcommand{\EstmcVarcond}[2]{\ensuremath{\Ley{#1\left|#2\right.}{\widehat{\varianza}}}\xspace}
%\newcommand{\EstmcCovcond}[2]{\ensuremath{\Ley{#1,#2}{\widehat{\covarianza}}}\xspace}


%%%% contrastación de hipótesis
\newcommand{\Hnula}{\ensuremath{H_0}\xspace}
\newcommand{\Halt}{\ensuremath{H_1}\xspace}
\newcommand{\Rcritica}{\ensuremath{RC}\xspace}
\newcommand{\Racept}{\ensuremath{RA}\xspace} % RA
\newcommand{\simnula}{\mathop{\sim}\limits_{\Hnula}}
\newcommand{\asimnula}{\mathop{\sim}\limits^{a}_{\Hnula}}

\newcommand{\testad}{\ensuremath{\mathcal{T}}\xspace}
%\newcommand{\testadistico}{\ensuremath{{\Estmc{\testad}}}\xspace}
%\newcommand{\testadistico}[1][{}]{\ensuremath{{\Estmc{\testad}}_{#1}}\xspace}
%\newcommand{\testadistico}{\ensuremath{{\Estm{\testad}{}}}\xspace}

%\newcommand{\testadistico}{\ensuremath{{\estm{\testad}{}}}\xspace}
\newcommand{\testadistico}{\ensuremath{\widehat{\testad}}\xspace}
%\newcommand{\testadistico}{\ensuremath{\testad}\xspace}

\newcommand{\Testadistico}{\ensuremath{{\Estm{\testad}{}}}\xspace}

\newcommand{\festad}{\ensuremath{\mathcal{F}}\xspace}
\newcommand{\festadistico}{\ensuremath{{\Estmc{\festad}}}\xspace}
\newcommand{\Festadistico}{\ensuremath{{\Estm{\festad}{}}}\xspace}

\newcommand{\IC}[3]{\ensuremath{\text{IC}_{#1}^{#2}(#3)}\xspace}

%%%%%%%%%%%%%%%%%% REGRESION
\newcommand{\ruido}{\per}
%\newcommand{\perturbacion}{{\ensuremath{\epsilon}}\xspace}
\newcommand{\perturbacion}{\ensuremath{u}\xspace}
\newcommand{\per}{\VA{\perturbacion}}
\newcommand{\peri}[1][\Ind]{\ensuremath{\VAi[{\color{\colorVA}#1}]{\perturbacion}}\xspace}


%% Notación para el regresor (MATRIZ)
\newcommand{\X}{\ensuremath{x}}

%% Con subíndices
\newcommand{\x}[1]{{\ensuremath{\X_{#1}}}\xspace}
\newcommand{\xk}{{\ensuremath{\X_{k}}}\xspace}
\newcommand{\Beta}[1]{{\ensuremath{\beta_{#1}}}\xspace}
\newcommand{\Betak}{{\ensuremath{\beta_{k}}}\xspace}

\newcommand{\DM}[1]{{\ensuremath{\ddot{#1}}}\xspace}
\newcommand{\ORT}[2]{{\ensuremath{{#1}_{\!{_{\!\bot}}{\!\!{\scriptscriptstyle \Mat{#2}\!}}}}}\xspace}
\newcommand{\NORT}[2]{{\ensuremath{{#1}_{\!\subset{\!\!{\scriptscriptstyle\Mat{#2}\!}}}}}\xspace}

%% Notación para el regresando (Vector)
\newcommand{\Y}{\ensuremath{y}}
%\newcommand{\Y}{\ensuremath{y}\xspace} % Notaci^^f3n para el regresando

%% término de error o innovación
\newcommand{\error}{\ensuremath{e}}

%% residuo o estimación del termino de error
\newcommand{\resc}{\ensuremath{{\Estmc{\error}}}\xspace}
\newcommand{\resic}[1]{\ensuremath{\Estmc{{\error}_{#1}}}\xspace}
%% residuo o estimación del termino de error
\newcommand{\resd}{\ensuremath{{\Estm{\res}{}}}\xspace}
\newcommand{\resid}[1]{\ensuremath{\Estm{{\error}_{#1}}{}}\xspace}

\newcommand{\res}{\resc}
\newcommand{\resi}[1]{\resic{#1}}

%% ESTIMADORES Y ESTIMACIONES (NOTACIÓN)

%% Estimador de la varianza de #1 que depende del regresando
\newcommand{\EstmdVAR}[1]{\EstmVAR{#1}{\Mat{\Y}}}
%% Estimador de la desviación típica de #1 que depende del regresando
\newcommand{\EstmdDT}[1]{\EstmDT{#1}{\Mat{\Y}}} 

%% Estimacion de la varianza de #1 
\newcommand{\EstmcVAR}[1]{\EstmVAR{#1}{}}
%% Estimacion de la desviación típica de #1  
\newcommand{\EstmcDT}[1]{\EstmDT{#1}{}} 

%% Estimador MCO de beta
\newcommand{\EstmdMCO}{\Vect{\Estmd{\beta}}}
%% Estimación MCO de beta
\newcommand{\EstmcMCO}{\Vect{\Estmc{\beta}}}
%% Estimador MCO de beta modificado (^#1}
\newcommand{\EstmdMCOm}[1]{\Vect{\Estmd{\beta^{#1}}}\xspace}
%% Estimación MCO de beta modificado (^#1}
\newcommand{\EstmcMCOm}[1]{\Vect{\Estmc{\beta^{#1}}}\xspace}

%% Estimador MCO de la varianza (sigma^2) 
\newcommand{\EstmdMCOVAR}{\ensuremath{\Estmd{\sigma^{2}}}\xspace}
%% Estimación MCO de la varianza (sigma^2) 
\newcommand{\EstmcMCOVAR}{\ensuremath{\Estmc{\sigma^{2}}}\xspace}

%% Estimador MCO de la desviación típica (sigma) 
\newcommand{\EstmdMCODT}{\Estmd{\sigma}}
%% Estimación MCO de la desviación típica (sigma) 
\newcommand{\EstmcMCODT}{\Estmc{\sigma}}

%% Estimación de la probabilidad ) 
\newcommand{\Estmcprobab}{\widehat{\probab}}
\newcommand{\EstmcPrb}[1]{\ensuremath{\Estmcprobab\left(#1\right)}\xspace}
\newcommand{\EstmcPrbcond}[2]{\EstmcPrb{\cOND{#1}{#2}}}



%% OPERACIONES CON MATRICES

\DeclareMathOperator{\Traza}{\eng{traza}{trace}}
\DeclareMathOperator{\Rango}{\eng{rango}{rank}}
\DeclareMathOperator{\rg}{rg}

\newcommand{\rango}[1]{\ensuremath{\Rango\left(#1\right)}\xspace}  
\newcommand{\traza}[1]{\ensuremath{\Traza\left(#1\right)}\xspace}  

%\NewDocumentCommand\MiNj{O{} m O{} m}{\ensuremath{\Mat[#1]{#2}\Mat[#3]{#4}}\xspace}

\newcommand{\MN}[2]{\ensuremath{\Mat{#1}\Mat{#2}}\xspace}
\newcommand{\MTN}[2]{\ensuremath{\MatT{#1}\Mat{#2}}\xspace}
\newcommand{\MTNT}[2]{\ensuremath{\MatT{#1}\MatT{#2}}\xspace}
\newcommand{\MTM}[1]{\MTN{#1}{#1}}
\newcommand{\MNT}[2]{\ensuremath{\Mat{#1}\MatT{#2}}\xspace}
\newcommand{\MMT}[1]{\ensuremath{\MNT{#1}{#1}}\xspace}
\newcommand{\MTMV}[2]{\ensuremath{\MTN{#1}{#1}\Vect{#2}}\xspace}

\newcommand{\VMW}[3]{\ensuremath{\VM{#1}{#2}\Vect{#3}}\xspace}

\newcommand{\MNMT}[2]{\ensuremath{\Mat{#1}\Mat{#2}\MatT{#1}}\xspace}

%% OPERACIONES CON VECTORES
%\newcommand{\ViTWjCC}[4]{\ensuremath{\VectC[#2]{#1}^{\!\T}\VectC[#4]{#3}}\xspace}

\newcommand{\ViTWj}[4]{\ensuremath{{\Vect[#2]{#1}}\Vect[#4]{#3}}\xspace}
\newcommand{\ViTVj}[3]{\ensuremath{{\Vect[#2]{#1}}\Vect[#3]{#1}}\xspace}

\renewcommand{\ViTWj}[4]{\ensuremath{\VectT[#2]{#1}\!\!\MVect[#4]{#3}}\xspace}
\renewcommand{\ViTVj}[3]{\ViTWj{#1}{#2}{#1}{#3}}

\newcommand{\VTW}[2]{\ViTWj{#1}{}{#2}{}}

%\newcommand{\VTW}[2]{\ensuremath{\VectT{#1}\,\Vect{#2}}\xspace}
\newcommand{\VTWT}[2]{\ensuremath{\VectT{#1}\VectT{#2}}\xspace}
%\newcommand{\VTV}[1]{\VTW{#1}{#1}}

\newcommand{\VTV}[2][{}]{\ViTWj{#2}{#1}{#2}{#1}}

\newcommand{\ViWjT}[4]{\ensuremath{\Vect[#2]{#1}{\VectT[#4]{#3}}}\xspace}
\newcommand{\ViVjT}[3]{\ensuremath{\Vect[#2]{#1}{\VectT[#3]{#1}}}\xspace}
\newcommand{\ViViT}[2][{}]{\ensuremath{\Vect[#1]{#2}{\VectT[#1]{#2}}}\xspace}

%\newcommand{\VWT}[2]{\ViWjT{}{#1}{}{#2}} # debería ser algo 

\newcommand{\VWT}[2]{\ensuremath{\MVect{#1}\!\VectT{#2}}\xspace}

\newcommand{\VVT}[1]{\ensuremath{\VWT{#1}{#1}}\xspace}
\newcommand{\VFW}[3][{}]{\ensuremath{\VectF[#1]{#2}\Vect{#3}}\xspace}

\newcommand{\esc}[2]{\ensuremath{\langle\Vect{#1},\Vect{#2}\rangle}}
\newcommand{\eSc}[2]{\ensuremath{\left< {#1}\, , \,{#2} \vphantom{\big(} \right>}}
\newcommand{\dotProd}[2]{\ensuremath{{#1}\cdot{#2}}}
\newcommand{\dotprod}[2]{\ensuremath{\Vect{#1}\cdot\Vect{#2}}}

\makeatletter
\newcommand*\bigcdot{\mathpalette\bigcdot@{.5}}
\newcommand*\bigcdot@[2]{\mathbin{\vcenter{\hbox{\scalebox{#2}{$\m@th#1\bullet$}}}}}
\makeatother
%%\newcommand{\dotprod}[2]{\ensuremath{\Vect{#1}\bigcdot\Vect{#2}}}



%% OPERACIONES CON MATRICES Y VECTORES
\newcommand{\VTM}[2]{\ensuremath{\VectT{#1}\Mat{#2}}\xspace}
\newcommand{\VTMT}[2]{\ensuremath{\VectT{#1}\MatT{#2}}\xspace}
\newcommand{\MTV}[2]{\ensuremath{\MatT{#1}\Vect{#2}}\xspace}
\newcommand{\MTVT}[2]{\ensuremath{\MatT{#1}\Vect{#2}}\xspace}
\newcommand{\VMT}[2]{\ensuremath{\Vect{#1}\MatT{#2}}\xspace}
\newcommand{\MVT}[2]{\ensuremath{\Mat{#1}\VectT{#2}}\xspace}
\newcommand{\MV}[3][{}]{\ensuremath{\Mat{#2}\Vect[#1]{#3}}\xspace}%MatVect_i
%\newcommand{\MV}[2]{\ensuremath{\Mat{#1}\Vect{#2}}\xspace}
\newcommand{\VM}[2]{\ensuremath{\Vect{#1}\Mat{#2}}\xspace}
\newcommand{\VTMV}[2]{\ensuremath{\VectT{#2}\MV{#1}{#2}}\xspace}
%\newcommand{\VFM}[3][{}]{\ensuremath{\VectF[#1]{#2}\Mat{#3}}\xspace}

\newcommand{\VMV}[2]{\ensuremath{\Vect{#2}\MV{#1}{#2}}\xspace}



\newcommand{\MTNdim}[4]{\ensuremath{\mathop{\MTN{#1}{#2}}\limits_{{[#3\times #4]}}}\xspace}
\newcommand{\MTVdim}[4]{\ensuremath{\mathop{\MTV{#1}{#2}}\limits_{{[#3\times #4]}}}\xspace}
\newcommand{\VTMdim}[4]{\ensuremath{\mathop{\MTV{#1}{#2}}\limits_{{[#3\times #4]}}}\xspace}

\newcommand{\XTX}{\MTM{\X}}
\newcommand{\XTY}{\ensuremath{\MatT{\X}\Vect{\Y}}\xspace}

\def\minus{\hbox{-}}
%\newcommand{\InvMat}[1]{\ensuremath{{\Mat{#1}}^{\minus\!1}}\xspace}
\newcommand{\InvMat}[2][{}]{\ensuremath{{{\mat{#2}}^{\minus1}_{#1}}}\xspace} 
%\newcommand{\InvMatP}[1]{\ensuremath{(\Mat{#1})^{\minus\!1}}\xspace}
\newcommand{\InvMatP}[2][{}]{\ensuremath{\left({\mat{#2}_{#1}}\right)^{\minus1}}\xspace} 
%% Matriz #1 traspuesta
%\newcommand{\MatT}[1]{\ensuremath{{\Mat{#1}\!}^{\T}\,}\xspace} 
\newcommand{\InvMatT}[2][{}]{\ensuremath{{{\mat{#2}}^{\minus\T}_{#1}}}\xspace} 


\newcommand{\InvMTM}[1]{\ensuremath{(\MTM{#1})^{\minus1}}\xspace}
\newcommand{\InvXTX}{\ensuremath{\InvMTM{\X}}\xspace}
\newcommand{\MInvMTMMT}[1]{\ensuremath{\Mat{#1}\InvMTM{#1}\MatT{#1}}\xspace}

%% Matrices unitarias
\newcommand{\Umat}[1]{\ensuremath{\Dot{\mat{#1}}}\xspace} 
%\newcommand{\Umat}[1]{\ensuremath{\Dot{\Dot{\mat{#1}}}}\xspace} 
%\newcommand{\Umat}[1]{\ensuremath{\breve{\mat{#1}}}\xspace} 
\newcommand{\UMat}[2][{}]{\ensuremath{{{\Umat{#2}}_{#1}}}\xspace} 
\newcommand{\UMatT}[2][{}]{\ensuremath{{{\Umat{#2}}^\T_{#1}}}\xspace} 
\newcommand{\InvUMat}[2][{}]{\ensuremath{{{\Umat{#2}}^{\minus1}_{#1}}}\xspace} 
\newcommand{\InvUMatT}[2][{}]{\ensuremath{{{\Umat{#2}}^{\minus\T}_{#1}}}\xspace} 

%% Transformaciones de Gauss
\newcommand{\MatGC}[1]{\ensuremath{{{\Umat{G}}_{#1\triangleright}}}\xspace} 
\newcommand{\MatGF}[1]{\ensuremath{{{\Umat{G}}_{\blacktriangledown\!#1}}}\xspace} 
\newcommand{\InvMatGC}[1]{\ensuremath{{{\Umat{G}}^{\minus1}_{#1\triangleright}}}\xspace} 
\newcommand{\InvMatGF}[1]{\ensuremath{{{\Umat{G}}^{\minus1}_{\blacktriangledown#1}}}\xspace} 
\newcommand{\MatGCT}[1]{\ensuremath{{{\Umat{G}}^{\T}_{#1\triangleright}}}\xspace} 


\newcommand{\UTU}{\VTV{\ruido}}
\newcommand{\ETE}{\VTV{\res}}

%% Sistema de ecuaciones lineales
\newcommand{\SEL}[3]{\ensuremath{\MV{#1}{#2}=\Vect{#3}}\xspace}
\newcommand{\SELT}[3]{\ensuremath{\left(\MatT{#1}\right)\Vect{#2}=\Vect{#3}}\xspace}


%%% REGRESIÓN MCO (EXPRESIONES MATRICIALES)

%% Regresión MCO #1 sobre #2
\newcommand{\MCO}[2]{\ensuremath{\InvMTM{#2}\MTV{#2}{#1}}\xspace}
%% Estimador MCO del regresando sobre el regresor
\newcommand{\MCOd}{\MCO{\VA{\Y}}{\X}}  
%% Estimación MCO del regresando sobre el regresor
\newcommand{\MCOc}{\MCO{\Y}{\X}}  



%%% SUMAS DE CUADRADOS (EXPRESIONES MATRICIALES)

%% suma de residuos al cuadrado
\newcommand{\SRC}{\VTV{\res}}
%% suma total de cuadrados
\newcommand{\STC}{\ensuremath{\VTV{\Y}-\TM\media{\Y}^2}\xspace}
%% suma explicada de cuadrados
\newcommand{\SEC}
        {\ensuremath{\VTV{\Estmc{\Y}}
        +\TM\media{\Y}^2
        -2\TM\media{\Y}\media{\Estmc{\Y}}}
        \xspace}
%% suma explicada de cuadrados (cuando hay término constante)
\newcommand{\SECcte}{\ensuremath{\VTV{\Estmc{\Y}}-\TM\media{\Y}^2}\xspace}

%%% SUMAS DE CUADRADOS (NOTACIÓN)

%% Estimadores
\newcommand{\EstmdSRC}{\ensuremath{\Estm{\text{SRC}}{}}\xspace}
\newcommand{\EstmdSTC}{\ensuremath{\Estm{\text{STC}}{}}\xspace}
\newcommand{\EstmdSEC}{\ensuremath{\Estm{\text{SEC}}{}}\xspace}
%% Estimaciones
\newcommand{\EstmcSRC}{\ensuremath{\text{SRC}}\xspace}
\newcommand{\EstmcSTC}{\ensuremath{\text{STC}}\xspace}
\newcommand{\EstmcSEC}{\ensuremath{\text{SEC}}\xspace}

%% COEFICIENTE DE DETERMINACIÓN

% (NOTACIÓN)
\newcommand{\EstmdRC}{\ensuremath{\Estm{R^2}{}}\xspace}
\newcommand{\EstmcRC}{\ensuremath{R^2}\xspace}
% (EXPRESIÓN MATRICIAL)
\newcommand{\RC}{\ensuremath{1-\frac{\EstmdSRC}{\EstmdSTC}}\xspace}

%% PREDICCIÓN
\newcommand{\Predictor}[2]{\ensuremath{\Estm{#1}{#2}}\xspace}

%% ÍNDICES
\newcommand{\TM}{\ensuremath{N}\xspace}
\newcommand{\Ind}{\MakeLowercase{n}}

%% CONJUNTOS
%\newcommand{\Z}{\mathbb Z}
\newcommand{\F}{\mathbb F}
%\newcommand{\R}{\mathbb R}
%\newcommand{\C}{\mathbb C}

\newcommand{\N}[1][{}]{\ensuremath{{{\mathbb{N}}^{#1}}}\xspace} 
\newcommand{\Z}[1][{}]{\ensuremath{{{\mathbb{Z}}^{#1}}}\xspace} 
\newcommand{\Q}[1][{}]{\ensuremath{{{\mathbb{Q}}^{#1}}}\xspace} 
\newcommand{\R}[1][{}]{\ensuremath{{{\mathbb{R}}^{#1}}}\xspace} 
\newcommand{\C}[1][{}]{\ensuremath{{{\mathbb{C}}^{#1}}}\xspace} 

\newcommand{\unaryminus}{\!\!\scalebox{0.75}[1.0]{\( - \)}}
\def\minus{%
  \setbox0=\hbox{-}%
  \vcenter{%
    \hrule width\wd0 height \the\fontdimen8\textfont3%
  }%
}

\endinput
