\documentclass[12pt,a4paper]{article} % tipo de documento con tamaño de letra y de papel

\usepackage[spanish]{babel}   % multi-language support

\usepackage{nacal}

% \usepackage{amsmath} % es necesario para que pinte OK el proceso de eliminación


\usepackage{pythontex} % para usar Python en el documento
\pythontexcustomc{pyconsole}{from nacal import *}
\pythontexcustomc{py}{from nacal import *}

\usepackage{fancyvrb} % para mostrar el código de manera literal

\title{Ejemplo mínimo para usar NAcAL en un documento de \LaTeX{}}
\author{Marcos Bujosa}

\begin{document}
\maketitle
\section{Primeros pasos del asunto}

\subsection{Preámbulo}

En el preámbulo del documento de \LaTeX{} hay que cargar el paquete
\texttt{pythontex} e indicar a \texttt{pythontex} que debe usar el
módulo NAcAL\dots{} El preámbulo de este documento incluye lo
siguiente:
\begin{Verbatim}
  \usepackage{pythontex}
  \pythontexcustomc{pyconsole}{from nacal import *}
  \pythontexcustomc{py}{from nacal import *}
\end{Verbatim}

\subsection{Compilación}
Para compilar el documento es necesario dar varias pasadas. Como
mínimo debemos ejecutar estas tres órdenes
\begin{Verbatim}
  pdflatex ParaMisCompis.tex
  
  pythontex --interpreter python:python3 ParaMisCompis.tex
  
  pdflatex ParaMisCompis.tex
  
\end{Verbatim}
donde \texttt{ParaMisCompis.tex} es el nombre del fichero que queréis
compilar. Si vuestro documento tiene enlaces internos, índice de
contenidos, glosario, bibliografía, etc. hay que compilar más veces y
ejecutar más comandos para que todo esté OK. Como éste es un ejemplo
mínimo, con esos tres comandos es sufi.


\section{Uso de Python}
Lo mejor es mirar la documentación de \texttt{pythontex}\\
(\texttt{https://www.ctan.org/pkg/pythontex})
\medskip

Otra cosa importante es que hay que tener en cuenta que en Python los
espacios (o tabuladores) delante de los comandos son importantes. Por
tanto el código Python debe empezar al comienzo de la línea. Por
ejemplo: este código es OK
\begin{Verbatim}
C = Matrix([[1,1,2,4],[1,0,3,4]])  
\end{Verbatim}
pero este otro daría error
\begin{Verbatim}
  C = Matrix([[1,1,2,4],[1,0,3,4]])  
\end{Verbatim}
(nótese que hay dos espacios en blanco al inicio).  \medskip

El código se puede incluir en el texto con \verb+\py+. Por ejemplo
\begin{Verbatim}
Dos más dos son $\py{2+2}$.
\end{Verbatim}
genera el siguiente texto:
\begin{quote}
  Dos más dos son $\py{2+2}$.
\end{quote}

NAcAL define una representación \LaTeX{} para la mayoría de objetos
que implementa (sistemas, vectores, matrices, transformaciones
elementales, subespacios afines, etc.) pero Python solo usa dicha
representación con los Notebooks de Jupyter. Si no se indica
explícitamente, Python no usa la representación \LaTeX{}. Por tanto,
con
\begin{Verbatim}
Sea la matriz $\py{ repr(Matrix([[1,1,2,4],[1,0,3,4]])) }$.
\end{Verbatim}
obtenemos el texto
\begin{quote}
  Sea la matriz $\py{ repr(Matrix([[1,1,2,4],[1,0,3,4]])) }$.
\end{quote}
Que es correcto pero muy feo.
\medskip


Si queremos que las cosas se pinten de manera bonita hay que decir
explícitamente a Python que use la representación \LaTeX{} con el
comando \texttt{latex()}. Por ejemplo, el código:

\begin{Verbatim}
Sea la matriz $\py{ latex( Matrix([[1,1,2,4],[1,0,3,4]]) ) }$.
\end{Verbatim}
nos escribe
\begin{quote}
  Sea la matriz $\py{ latex( Matrix([[1,1,2,4],[1,0,3,4]]) ) }$.
\end{quote}
\medskip


Hay otra forma de trabajar (que es la que uso con más frecuencia). El
entorno \texttt{pycode} ejecuta código de Python pero no muestra el
resultado (hay que mirar la documentación de Pythontex).
\begin{pycode}
C = Matrix([[1,1,2,4],[1,0,3,4]])
\end{pycode}  
Así, con
\begin{Verbatim}
\begin{pycode}
C = Matrix([[1,1,2,4],[1,0,3,4]])
\end{pycode}  
\end{Verbatim}
ya hemos definido la matriz $\mathbf{C}$. Ahora podemos escribir
\begin{Verbatim}
  Sea la matriz $\py{latex( C )}$, que tiene rango $\py{C.rango()}$ y
  cuya forma escalonada por filas es $\py{ latex(C.U()) }$.
\end{Verbatim}
y obtener
\begin{quote}
  Sea la matriz $\py{latex( C )}$, que tiene rango $\py{C.rango()}$ y
  cuya forma escalonada por filas es $\py{ latex(C.U()) }$.
\end{quote}

\section{Uso de NAcAL}
Algunos procedimientos del curso (como la eliminación) dan como
resultado una matriz, pero también muestran como llegar al
resultado. Por ejemplo, podemos escalonar por columnas la matriz
$\mathbf{C}$
\begin{Verbatim}
  \begin{displaymath}
    \py{latex( Elim(C) ) }
  \end{displaymath}
\end{Verbatim}
\begin{displaymath}
  \py{latex( Elim(C) ) }
\end{displaymath}
Pero solo nos ha mostrado el resultado final (es decir, la matriz
escalonada por eliminación). Afortunadamente NAcAL guarda el código
\LaTeX{} de los pasos dados para llegar al resultado en el atributo
\texttt{.tex}. Como dicho atributo ya es código \LaTeX{}, no hay que
usar el comando \texttt{latex()} dentro de Python.
\begin{Verbatim}
  \begin{displaymath}
    \py{Elim(C).tex}
  \end{displaymath}
\end{Verbatim}
nos muestra lo siguiente
\begin{displaymath}
  \py{Elim(C).tex}
\end{displaymath}

Hay muchos objetos molestos para escribir en \LaTeX{} pero que
resultan inmediatos con NAcAL. Por ejemplo el conjunto de soluciones
$\mathbf{C}\boldsymbol{x}=\boldsymbol{0}$
\begin{Verbatim}
  \begin{displaymath}
    \py{ latex( C.espacio_nulo() ) }
  \end{displaymath}
\end{Verbatim}
nos muestra lo siguiente
\begin{displaymath}
  \py{latex(C.espacio_nulo())}
\end{displaymath}
Si solo estamos interesados en escribir las ecuaciones paramétricas
\begin{Verbatim}
  \begin{displaymath}
    \py{ C.espacio_nulo().EcParametricas() }
  \end{displaymath}
\end{Verbatim}
nos muestra lo siguiente
\begin{displaymath}
  \py{C.espacio_nulo().EcParametricas()}
\end{displaymath}
\DefineShortVerb{\|}
(escribiendo esto me doy cuenta de que no he sido sistemático con los
nombres\ldots{} \Verb+C.espacio_nulo().EcParametricas()+ ya es código
\LaTeX{}, y por el nombre no está claro\dots{} quizá debería darle una
vuelta a eso).  \medskip

En ocasiones he elegido otra representación para lo mismo. Por
ejemplo, la clase \texttt{Homogenea} resuelve el sistema de ecuaciones
homogéneo
\begin{Verbatim}
  \begin{displaymath}
    \py{ latex( Homogenea(C) )}
  \end{displaymath}
\end{Verbatim}
nos muestra lo siguiente
\begin{displaymath}
  \py{latex( Homogenea(C) )}
\end{displaymath}
Y si queremos ver cómo lo ha calculado\dots
\begin{Verbatim}
  \begin{displaymath}
    \py{Homogenea(C).tex }
  \end{displaymath}
\end{Verbatim}
nos muestra lo siguiente
\begin{displaymath}
  \py{Homogenea(C).tex }
\end{displaymath}
\bigskip

Como podéis ver, escribir con Pythontex + NAcAL es bastante
productivo, ahorra mucho trabajo y evita muchos
errores. \textbf{!Espero que os animéis a usarlo!}\dots{} que yo sepa
no hay nada igual

Como podéis imaginar la inversión de tiempo, trabajo y aprendizaje ha
sido bastante considerable\dots{} sería un desperdicio que solo lo use
yo.  \bigskip

Además, el código es abierto, así que podéis añadir lo que le falte.


\end{document}